\documentclass[11pt]{article}

    \usepackage[breakable]{tcolorbox}
    \usepackage{parskip} % Stop auto-indenting (to mimic markdown behaviour)
    

    % Basic figure setup, for now with no caption control since it's done
    % automatically by Pandoc (which extracts ![](path) syntax from Markdown).
    \usepackage{graphicx}
    % Keep aspect ratio if custom image width or height is specified
    \setkeys{Gin}{keepaspectratio}
    % Maintain compatibility with old templates. Remove in nbconvert 6.0
    \let\Oldincludegraphics\includegraphics
    % Ensure that by default, figures have no caption (until we provide a
    % proper Figure object with a Caption API and a way to capture that
    % in the conversion process - todo).
    \usepackage{caption}
    \DeclareCaptionFormat{nocaption}{}
    \captionsetup{format=nocaption,aboveskip=0pt,belowskip=0pt}

    \usepackage{float}
    \floatplacement{figure}{H} % forces figures to be placed at the correct location
    \usepackage{xcolor} % Allow colors to be defined
    \usepackage{enumerate} % Needed for markdown enumerations to work
    \usepackage{geometry} % Used to adjust the document margins
    \usepackage{amsmath} % Equations
    \usepackage{amssymb} % Equations
    \usepackage{textcomp} % defines textquotesingle
    % Hack from http://tex.stackexchange.com/a/47451/13684:
    \AtBeginDocument{%
        \def\PYZsq{\textquotesingle}% Upright quotes in Pygmentized code
    }
    \usepackage{upquote} % Upright quotes for verbatim code
    \usepackage{eurosym} % defines \euro

    \usepackage{iftex}
    \ifPDFTeX
        \usepackage[T1]{fontenc}
        \IfFileExists{alphabeta.sty}{
              \usepackage{alphabeta}
          }{
              \usepackage[mathletters]{ucs}
              \usepackage[utf8x]{inputenc}
          }
    \else
        \usepackage{fontspec}
        \usepackage{unicode-math}
    \fi

    \usepackage{fancyvrb} % verbatim replacement that allows latex
    \usepackage{grffile} % extends the file name processing of package graphics
                         % to support a larger range
    \makeatletter % fix for old versions of grffile with XeLaTeX
    \@ifpackagelater{grffile}{2019/11/01}
    {
      % Do nothing on new versions
    }
    {
      \def\Gread@@xetex#1{%
        \IfFileExists{"\Gin@base".bb}%
        {\Gread@eps{\Gin@base.bb}}%
        {\Gread@@xetex@aux#1}%
      }
    }
    \makeatother
    \usepackage[Export]{adjustbox} % Used to constrain images to a maximum size
    \adjustboxset{max size={0.9\linewidth}{0.9\paperheight}}

    % The hyperref package gives us a pdf with properly built
    % internal navigation ('pdf bookmarks' for the table of contents,
    % internal cross-reference links, web links for URLs, etc.)
    \usepackage{hyperref}
    % The default LaTeX title has an obnoxious amount of whitespace. By default,
    % titling removes some of it. It also provides customization options.
    \usepackage{titling}
    \usepackage{longtable} % longtable support required by pandoc >1.10
    \usepackage{booktabs}  % table support for pandoc > 1.12.2
    \usepackage{array}     % table support for pandoc >= 2.11.3
    \usepackage{calc}      % table minipage width calculation for pandoc >= 2.11.1
    \usepackage[inline]{enumitem} % IRkernel/repr support (it uses the enumerate* environment)
    \usepackage[normalem]{ulem} % ulem is needed to support strikethroughs (\sout)
                                % normalem makes italics be italics, not underlines
    \usepackage{soul}      % strikethrough (\st) support for pandoc >= 3.0.0
    \usepackage{mathrsfs}
    

    
    % Colors for the hyperref package
    \definecolor{urlcolor}{rgb}{0,.145,.698}
    \definecolor{linkcolor}{rgb}{.71,0.21,0.01}
    \definecolor{citecolor}{rgb}{.12,.54,.11}

    % ANSI colors
    \definecolor{ansi-black}{HTML}{3E424D}
    \definecolor{ansi-black-intense}{HTML}{282C36}
    \definecolor{ansi-red}{HTML}{E75C58}
    \definecolor{ansi-red-intense}{HTML}{B22B31}
    \definecolor{ansi-green}{HTML}{00A250}
    \definecolor{ansi-green-intense}{HTML}{007427}
    \definecolor{ansi-yellow}{HTML}{DDB62B}
    \definecolor{ansi-yellow-intense}{HTML}{B27D12}
    \definecolor{ansi-blue}{HTML}{208FFB}
    \definecolor{ansi-blue-intense}{HTML}{0065CA}
    \definecolor{ansi-magenta}{HTML}{D160C4}
    \definecolor{ansi-magenta-intense}{HTML}{A03196}
    \definecolor{ansi-cyan}{HTML}{60C6C8}
    \definecolor{ansi-cyan-intense}{HTML}{258F8F}
    \definecolor{ansi-white}{HTML}{C5C1B4}
    \definecolor{ansi-white-intense}{HTML}{A1A6B2}
    \definecolor{ansi-default-inverse-fg}{HTML}{FFFFFF}
    \definecolor{ansi-default-inverse-bg}{HTML}{000000}

    % common color for the border for error outputs.
    \definecolor{outerrorbackground}{HTML}{FFDFDF}

    % commands and environments needed by pandoc snippets
    % extracted from the output of `pandoc -s`
    \providecommand{\tightlist}{%
      \setlength{\itemsep}{0pt}\setlength{\parskip}{0pt}}
    \DefineVerbatimEnvironment{Highlighting}{Verbatim}{commandchars=\\\{\}}
    % Add ',fontsize=\small' for more characters per line
    \newenvironment{Shaded}{}{}
    \newcommand{\KeywordTok}[1]{\textcolor[rgb]{0.00,0.44,0.13}{\textbf{{#1}}}}
    \newcommand{\DataTypeTok}[1]{\textcolor[rgb]{0.56,0.13,0.00}{{#1}}}
    \newcommand{\DecValTok}[1]{\textcolor[rgb]{0.25,0.63,0.44}{{#1}}}
    \newcommand{\BaseNTok}[1]{\textcolor[rgb]{0.25,0.63,0.44}{{#1}}}
    \newcommand{\FloatTok}[1]{\textcolor[rgb]{0.25,0.63,0.44}{{#1}}}
    \newcommand{\CharTok}[1]{\textcolor[rgb]{0.25,0.44,0.63}{{#1}}}
    \newcommand{\StringTok}[1]{\textcolor[rgb]{0.25,0.44,0.63}{{#1}}}
    \newcommand{\CommentTok}[1]{\textcolor[rgb]{0.38,0.63,0.69}{\textit{{#1}}}}
    \newcommand{\OtherTok}[1]{\textcolor[rgb]{0.00,0.44,0.13}{{#1}}}
    \newcommand{\AlertTok}[1]{\textcolor[rgb]{1.00,0.00,0.00}{\textbf{{#1}}}}
    \newcommand{\FunctionTok}[1]{\textcolor[rgb]{0.02,0.16,0.49}{{#1}}}
    \newcommand{\RegionMarkerTok}[1]{{#1}}
    \newcommand{\ErrorTok}[1]{\textcolor[rgb]{1.00,0.00,0.00}{\textbf{{#1}}}}
    \newcommand{\NormalTok}[1]{{#1}}

    % Additional commands for more recent versions of Pandoc
    \newcommand{\ConstantTok}[1]{\textcolor[rgb]{0.53,0.00,0.00}{{#1}}}
    \newcommand{\SpecialCharTok}[1]{\textcolor[rgb]{0.25,0.44,0.63}{{#1}}}
    \newcommand{\VerbatimStringTok}[1]{\textcolor[rgb]{0.25,0.44,0.63}{{#1}}}
    \newcommand{\SpecialStringTok}[1]{\textcolor[rgb]{0.73,0.40,0.53}{{#1}}}
    \newcommand{\ImportTok}[1]{{#1}}
    \newcommand{\DocumentationTok}[1]{\textcolor[rgb]{0.73,0.13,0.13}{\textit{{#1}}}}
    \newcommand{\AnnotationTok}[1]{\textcolor[rgb]{0.38,0.63,0.69}{\textbf{\textit{{#1}}}}}
    \newcommand{\CommentVarTok}[1]{\textcolor[rgb]{0.38,0.63,0.69}{\textbf{\textit{{#1}}}}}
    \newcommand{\VariableTok}[1]{\textcolor[rgb]{0.10,0.09,0.49}{{#1}}}
    \newcommand{\ControlFlowTok}[1]{\textcolor[rgb]{0.00,0.44,0.13}{\textbf{{#1}}}}
    \newcommand{\OperatorTok}[1]{\textcolor[rgb]{0.40,0.40,0.40}{{#1}}}
    \newcommand{\BuiltInTok}[1]{{#1}}
    \newcommand{\ExtensionTok}[1]{{#1}}
    \newcommand{\PreprocessorTok}[1]{\textcolor[rgb]{0.74,0.48,0.00}{{#1}}}
    \newcommand{\AttributeTok}[1]{\textcolor[rgb]{0.49,0.56,0.16}{{#1}}}
    \newcommand{\InformationTok}[1]{\textcolor[rgb]{0.38,0.63,0.69}{\textbf{\textit{{#1}}}}}
    \newcommand{\WarningTok}[1]{\textcolor[rgb]{0.38,0.63,0.69}{\textbf{\textit{{#1}}}}}
    \makeatletter
    \newsavebox\pandoc@box
    \newcommand*\pandocbounded[1]{%
      \sbox\pandoc@box{#1}%
      % scaling factors for width and height
      \Gscale@div\@tempa\textheight{\dimexpr\ht\pandoc@box+\dp\pandoc@box\relax}%
      \Gscale@div\@tempb\linewidth{\wd\pandoc@box}%
      % select the smaller of both
      \ifdim\@tempb\p@<\@tempa\p@
        \let\@tempa\@tempb
      \fi
      % scaling accordingly (\@tempa < 1)
      \ifdim\@tempa\p@<\p@
        \scalebox{\@tempa}{\usebox\pandoc@box}%
      % scaling not needed, use as it is
      \else
        \usebox{\pandoc@box}%
      \fi
    }
    \makeatother

    % Define a nice break command that doesn't care if a line doesn't already
    % exist.
    \def\br{\hspace*{\fill} \\* }
    % Math Jax compatibility definitions
    \def\gt{>}
    \def\lt{<}
    \let\Oldtex\TeX
    \let\Oldlatex\LaTeX
    \renewcommand{\TeX}{\textrm{\Oldtex}}
    \renewcommand{\LaTeX}{\textrm{\Oldlatex}}
    % Document parameters
    % Document title
    \title{Analiza\_mieszkan\_Final}
    
    
    
    
    
    
    
% Pygments definitions
\makeatletter
\def\PY@reset{\let\PY@it=\relax \let\PY@bf=\relax%
    \let\PY@ul=\relax \let\PY@tc=\relax%
    \let\PY@bc=\relax \let\PY@ff=\relax}
\def\PY@tok#1{\csname PY@tok@#1\endcsname}
\def\PY@toks#1+{\ifx\relax#1\empty\else%
    \PY@tok{#1}\expandafter\PY@toks\fi}
\def\PY@do#1{\PY@bc{\PY@tc{\PY@ul{%
    \PY@it{\PY@bf{\PY@ff{#1}}}}}}}
\def\PY#1#2{\PY@reset\PY@toks#1+\relax+\PY@do{#2}}

\@namedef{PY@tok@w}{\def\PY@tc##1{\textcolor[rgb]{0.73,0.73,0.73}{##1}}}
\@namedef{PY@tok@c}{\let\PY@it=\textit\def\PY@tc##1{\textcolor[rgb]{0.24,0.48,0.48}{##1}}}
\@namedef{PY@tok@cp}{\def\PY@tc##1{\textcolor[rgb]{0.61,0.40,0.00}{##1}}}
\@namedef{PY@tok@k}{\let\PY@bf=\textbf\def\PY@tc##1{\textcolor[rgb]{0.00,0.50,0.00}{##1}}}
\@namedef{PY@tok@kp}{\def\PY@tc##1{\textcolor[rgb]{0.00,0.50,0.00}{##1}}}
\@namedef{PY@tok@kt}{\def\PY@tc##1{\textcolor[rgb]{0.69,0.00,0.25}{##1}}}
\@namedef{PY@tok@o}{\def\PY@tc##1{\textcolor[rgb]{0.40,0.40,0.40}{##1}}}
\@namedef{PY@tok@ow}{\let\PY@bf=\textbf\def\PY@tc##1{\textcolor[rgb]{0.67,0.13,1.00}{##1}}}
\@namedef{PY@tok@nb}{\def\PY@tc##1{\textcolor[rgb]{0.00,0.50,0.00}{##1}}}
\@namedef{PY@tok@nf}{\def\PY@tc##1{\textcolor[rgb]{0.00,0.00,1.00}{##1}}}
\@namedef{PY@tok@nc}{\let\PY@bf=\textbf\def\PY@tc##1{\textcolor[rgb]{0.00,0.00,1.00}{##1}}}
\@namedef{PY@tok@nn}{\let\PY@bf=\textbf\def\PY@tc##1{\textcolor[rgb]{0.00,0.00,1.00}{##1}}}
\@namedef{PY@tok@ne}{\let\PY@bf=\textbf\def\PY@tc##1{\textcolor[rgb]{0.80,0.25,0.22}{##1}}}
\@namedef{PY@tok@nv}{\def\PY@tc##1{\textcolor[rgb]{0.10,0.09,0.49}{##1}}}
\@namedef{PY@tok@no}{\def\PY@tc##1{\textcolor[rgb]{0.53,0.00,0.00}{##1}}}
\@namedef{PY@tok@nl}{\def\PY@tc##1{\textcolor[rgb]{0.46,0.46,0.00}{##1}}}
\@namedef{PY@tok@ni}{\let\PY@bf=\textbf\def\PY@tc##1{\textcolor[rgb]{0.44,0.44,0.44}{##1}}}
\@namedef{PY@tok@na}{\def\PY@tc##1{\textcolor[rgb]{0.41,0.47,0.13}{##1}}}
\@namedef{PY@tok@nt}{\let\PY@bf=\textbf\def\PY@tc##1{\textcolor[rgb]{0.00,0.50,0.00}{##1}}}
\@namedef{PY@tok@nd}{\def\PY@tc##1{\textcolor[rgb]{0.67,0.13,1.00}{##1}}}
\@namedef{PY@tok@s}{\def\PY@tc##1{\textcolor[rgb]{0.73,0.13,0.13}{##1}}}
\@namedef{PY@tok@sd}{\let\PY@it=\textit\def\PY@tc##1{\textcolor[rgb]{0.73,0.13,0.13}{##1}}}
\@namedef{PY@tok@si}{\let\PY@bf=\textbf\def\PY@tc##1{\textcolor[rgb]{0.64,0.35,0.47}{##1}}}
\@namedef{PY@tok@se}{\let\PY@bf=\textbf\def\PY@tc##1{\textcolor[rgb]{0.67,0.36,0.12}{##1}}}
\@namedef{PY@tok@sr}{\def\PY@tc##1{\textcolor[rgb]{0.64,0.35,0.47}{##1}}}
\@namedef{PY@tok@ss}{\def\PY@tc##1{\textcolor[rgb]{0.10,0.09,0.49}{##1}}}
\@namedef{PY@tok@sx}{\def\PY@tc##1{\textcolor[rgb]{0.00,0.50,0.00}{##1}}}
\@namedef{PY@tok@m}{\def\PY@tc##1{\textcolor[rgb]{0.40,0.40,0.40}{##1}}}
\@namedef{PY@tok@gh}{\let\PY@bf=\textbf\def\PY@tc##1{\textcolor[rgb]{0.00,0.00,0.50}{##1}}}
\@namedef{PY@tok@gu}{\let\PY@bf=\textbf\def\PY@tc##1{\textcolor[rgb]{0.50,0.00,0.50}{##1}}}
\@namedef{PY@tok@gd}{\def\PY@tc##1{\textcolor[rgb]{0.63,0.00,0.00}{##1}}}
\@namedef{PY@tok@gi}{\def\PY@tc##1{\textcolor[rgb]{0.00,0.52,0.00}{##1}}}
\@namedef{PY@tok@gr}{\def\PY@tc##1{\textcolor[rgb]{0.89,0.00,0.00}{##1}}}
\@namedef{PY@tok@ge}{\let\PY@it=\textit}
\@namedef{PY@tok@gs}{\let\PY@bf=\textbf}
\@namedef{PY@tok@ges}{\let\PY@bf=\textbf\let\PY@it=\textit}
\@namedef{PY@tok@gp}{\let\PY@bf=\textbf\def\PY@tc##1{\textcolor[rgb]{0.00,0.00,0.50}{##1}}}
\@namedef{PY@tok@go}{\def\PY@tc##1{\textcolor[rgb]{0.44,0.44,0.44}{##1}}}
\@namedef{PY@tok@gt}{\def\PY@tc##1{\textcolor[rgb]{0.00,0.27,0.87}{##1}}}
\@namedef{PY@tok@err}{\def\PY@bc##1{{\setlength{\fboxsep}{\string -\fboxrule}\fcolorbox[rgb]{1.00,0.00,0.00}{1,1,1}{\strut ##1}}}}
\@namedef{PY@tok@kc}{\let\PY@bf=\textbf\def\PY@tc##1{\textcolor[rgb]{0.00,0.50,0.00}{##1}}}
\@namedef{PY@tok@kd}{\let\PY@bf=\textbf\def\PY@tc##1{\textcolor[rgb]{0.00,0.50,0.00}{##1}}}
\@namedef{PY@tok@kn}{\let\PY@bf=\textbf\def\PY@tc##1{\textcolor[rgb]{0.00,0.50,0.00}{##1}}}
\@namedef{PY@tok@kr}{\let\PY@bf=\textbf\def\PY@tc##1{\textcolor[rgb]{0.00,0.50,0.00}{##1}}}
\@namedef{PY@tok@bp}{\def\PY@tc##1{\textcolor[rgb]{0.00,0.50,0.00}{##1}}}
\@namedef{PY@tok@fm}{\def\PY@tc##1{\textcolor[rgb]{0.00,0.00,1.00}{##1}}}
\@namedef{PY@tok@vc}{\def\PY@tc##1{\textcolor[rgb]{0.10,0.09,0.49}{##1}}}
\@namedef{PY@tok@vg}{\def\PY@tc##1{\textcolor[rgb]{0.10,0.09,0.49}{##1}}}
\@namedef{PY@tok@vi}{\def\PY@tc##1{\textcolor[rgb]{0.10,0.09,0.49}{##1}}}
\@namedef{PY@tok@vm}{\def\PY@tc##1{\textcolor[rgb]{0.10,0.09,0.49}{##1}}}
\@namedef{PY@tok@sa}{\def\PY@tc##1{\textcolor[rgb]{0.73,0.13,0.13}{##1}}}
\@namedef{PY@tok@sb}{\def\PY@tc##1{\textcolor[rgb]{0.73,0.13,0.13}{##1}}}
\@namedef{PY@tok@sc}{\def\PY@tc##1{\textcolor[rgb]{0.73,0.13,0.13}{##1}}}
\@namedef{PY@tok@dl}{\def\PY@tc##1{\textcolor[rgb]{0.73,0.13,0.13}{##1}}}
\@namedef{PY@tok@s2}{\def\PY@tc##1{\textcolor[rgb]{0.73,0.13,0.13}{##1}}}
\@namedef{PY@tok@sh}{\def\PY@tc##1{\textcolor[rgb]{0.73,0.13,0.13}{##1}}}
\@namedef{PY@tok@s1}{\def\PY@tc##1{\textcolor[rgb]{0.73,0.13,0.13}{##1}}}
\@namedef{PY@tok@mb}{\def\PY@tc##1{\textcolor[rgb]{0.40,0.40,0.40}{##1}}}
\@namedef{PY@tok@mf}{\def\PY@tc##1{\textcolor[rgb]{0.40,0.40,0.40}{##1}}}
\@namedef{PY@tok@mh}{\def\PY@tc##1{\textcolor[rgb]{0.40,0.40,0.40}{##1}}}
\@namedef{PY@tok@mi}{\def\PY@tc##1{\textcolor[rgb]{0.40,0.40,0.40}{##1}}}
\@namedef{PY@tok@il}{\def\PY@tc##1{\textcolor[rgb]{0.40,0.40,0.40}{##1}}}
\@namedef{PY@tok@mo}{\def\PY@tc##1{\textcolor[rgb]{0.40,0.40,0.40}{##1}}}
\@namedef{PY@tok@ch}{\let\PY@it=\textit\def\PY@tc##1{\textcolor[rgb]{0.24,0.48,0.48}{##1}}}
\@namedef{PY@tok@cm}{\let\PY@it=\textit\def\PY@tc##1{\textcolor[rgb]{0.24,0.48,0.48}{##1}}}
\@namedef{PY@tok@cpf}{\let\PY@it=\textit\def\PY@tc##1{\textcolor[rgb]{0.24,0.48,0.48}{##1}}}
\@namedef{PY@tok@c1}{\let\PY@it=\textit\def\PY@tc##1{\textcolor[rgb]{0.24,0.48,0.48}{##1}}}
\@namedef{PY@tok@cs}{\let\PY@it=\textit\def\PY@tc##1{\textcolor[rgb]{0.24,0.48,0.48}{##1}}}

\def\PYZbs{\char`\\}
\def\PYZus{\char`\_}
\def\PYZob{\char`\{}
\def\PYZcb{\char`\}}
\def\PYZca{\char`\^}
\def\PYZam{\char`\&}
\def\PYZlt{\char`\<}
\def\PYZgt{\char`\>}
\def\PYZsh{\char`\#}
\def\PYZpc{\char`\%}
\def\PYZdl{\char`\$}
\def\PYZhy{\char`\-}
\def\PYZsq{\char`\'}
\def\PYZdq{\char`\"}
\def\PYZti{\char`\~}
% for compatibility with earlier versions
\def\PYZat{@}
\def\PYZlb{[}
\def\PYZrb{]}
\makeatother


    % For linebreaks inside Verbatim environment from package fancyvrb.
    \makeatletter
        \newbox\Wrappedcontinuationbox
        \newbox\Wrappedvisiblespacebox
        \newcommand*\Wrappedvisiblespace {\textcolor{red}{\textvisiblespace}}
        \newcommand*\Wrappedcontinuationsymbol {\textcolor{red}{\llap{\tiny$\m@th\hookrightarrow$}}}
        \newcommand*\Wrappedcontinuationindent {3ex }
        \newcommand*\Wrappedafterbreak {\kern\Wrappedcontinuationindent\copy\Wrappedcontinuationbox}
        % Take advantage of the already applied Pygments mark-up to insert
        % potential linebreaks for TeX processing.
        %        {, <, #, %, $, ' and ": go to next line.
        %        _, }, ^, &, >, - and ~: stay at end of broken line.
        % Use of \textquotesingle for straight quote.
        \newcommand*\Wrappedbreaksatspecials {%
            \def\PYGZus{\discretionary{\char`\_}{\Wrappedafterbreak}{\char`\_}}%
            \def\PYGZob{\discretionary{}{\Wrappedafterbreak\char`\{}{\char`\{}}%
            \def\PYGZcb{\discretionary{\char`\}}{\Wrappedafterbreak}{\char`\}}}%
            \def\PYGZca{\discretionary{\char`\^}{\Wrappedafterbreak}{\char`\^}}%
            \def\PYGZam{\discretionary{\char`\&}{\Wrappedafterbreak}{\char`\&}}%
            \def\PYGZlt{\discretionary{}{\Wrappedafterbreak\char`\<}{\char`\<}}%
            \def\PYGZgt{\discretionary{\char`\>}{\Wrappedafterbreak}{\char`\>}}%
            \def\PYGZsh{\discretionary{}{\Wrappedafterbreak\char`\#}{\char`\#}}%
            \def\PYGZpc{\discretionary{}{\Wrappedafterbreak\char`\%}{\char`\%}}%
            \def\PYGZdl{\discretionary{}{\Wrappedafterbreak\char`\$}{\char`\$}}%
            \def\PYGZhy{\discretionary{\char`\-}{\Wrappedafterbreak}{\char`\-}}%
            \def\PYGZsq{\discretionary{}{\Wrappedafterbreak\textquotesingle}{\textquotesingle}}%
            \def\PYGZdq{\discretionary{}{\Wrappedafterbreak\char`\"}{\char`\"}}%
            \def\PYGZti{\discretionary{\char`\~}{\Wrappedafterbreak}{\char`\~}}%
        }
        % Some characters . , ; ? ! / are not pygmentized.
        % This macro makes them "active" and they will insert potential linebreaks
        \newcommand*\Wrappedbreaksatpunct {%
            \lccode`\~`\.\lowercase{\def~}{\discretionary{\hbox{\char`\.}}{\Wrappedafterbreak}{\hbox{\char`\.}}}%
            \lccode`\~`\,\lowercase{\def~}{\discretionary{\hbox{\char`\,}}{\Wrappedafterbreak}{\hbox{\char`\,}}}%
            \lccode`\~`\;\lowercase{\def~}{\discretionary{\hbox{\char`\;}}{\Wrappedafterbreak}{\hbox{\char`\;}}}%
            \lccode`\~`\:\lowercase{\def~}{\discretionary{\hbox{\char`\:}}{\Wrappedafterbreak}{\hbox{\char`\:}}}%
            \lccode`\~`\?\lowercase{\def~}{\discretionary{\hbox{\char`\?}}{\Wrappedafterbreak}{\hbox{\char`\?}}}%
            \lccode`\~`\!\lowercase{\def~}{\discretionary{\hbox{\char`\!}}{\Wrappedafterbreak}{\hbox{\char`\!}}}%
            \lccode`\~`\/\lowercase{\def~}{\discretionary{\hbox{\char`\/}}{\Wrappedafterbreak}{\hbox{\char`\/}}}%
            \catcode`\.\active
            \catcode`\,\active
            \catcode`\;\active
            \catcode`\:\active
            \catcode`\?\active
            \catcode`\!\active
            \catcode`\/\active
            \lccode`\~`\~
        }
    \makeatother

    \let\OriginalVerbatim=\Verbatim
    \makeatletter
    \renewcommand{\Verbatim}[1][1]{%
        %\parskip\z@skip
        \sbox\Wrappedcontinuationbox {\Wrappedcontinuationsymbol}%
        \sbox\Wrappedvisiblespacebox {\FV@SetupFont\Wrappedvisiblespace}%
        \def\FancyVerbFormatLine ##1{\hsize\linewidth
            \vtop{\raggedright\hyphenpenalty\z@\exhyphenpenalty\z@
                \doublehyphendemerits\z@\finalhyphendemerits\z@
                \strut ##1\strut}%
        }%
        % If the linebreak is at a space, the latter will be displayed as visible
        % space at end of first line, and a continuation symbol starts next line.
        % Stretch/shrink are however usually zero for typewriter font.
        \def\FV@Space {%
            \nobreak\hskip\z@ plus\fontdimen3\font minus\fontdimen4\font
            \discretionary{\copy\Wrappedvisiblespacebox}{\Wrappedafterbreak}
            {\kern\fontdimen2\font}%
        }%

        % Allow breaks at special characters using \PYG... macros.
        \Wrappedbreaksatspecials
        % Breaks at punctuation characters . , ; ? ! and / need catcode=\active
        \OriginalVerbatim[#1,codes*=\Wrappedbreaksatpunct]%
    }
    \makeatother

    % Exact colors from NB
    \definecolor{incolor}{HTML}{303F9F}
    \definecolor{outcolor}{HTML}{D84315}
    \definecolor{cellborder}{HTML}{CFCFCF}
    \definecolor{cellbackground}{HTML}{F7F7F7}

    % prompt
    \makeatletter
    \newcommand{\boxspacing}{\kern\kvtcb@left@rule\kern\kvtcb@boxsep}
    \makeatother
    \newcommand{\prompt}[4]{
        {\ttfamily\llap{{\color{#2}[#3]:\hspace{3pt}#4}}\vspace{-\baselineskip}}
    }
    

    
    % Prevent overflowing lines due to hard-to-break entities
    \sloppy
    % Setup hyperref package
    \hypersetup{
      breaklinks=true,  % so long urls are correctly broken across lines
      colorlinks=true,
      urlcolor=urlcolor,
      linkcolor=linkcolor,
      citecolor=citecolor,
      }
    % Slightly bigger margins than the latex defaults
    
    \geometry{verbose,tmargin=1in,bmargin=1in,lmargin=1in,rmargin=1in}
    
    

\begin{document}
    
    \maketitle
    
    

    
    \section{WPROWADZENIE}\label{wprowadzenie}

Wykorzystywany zbiór danych pochodzi z serwisu
\href{https://www.kaggle.com/datasets/krzysztofjamroz/apartment-prices-in-poland/data}{Kaggle.com}
i zawiera ceny mieszkań z 15 największych miast w Polsce. Rzeczone dane
były zbierane co miesiąc, od sierpnia 2023 do czerwca 2024.

Po pobraniu paczki spakowanych danych wyłaniają się przed nami pliki o
następującej strukturze: * apartments\_pl\_YYYY\_MM.csv *
apartments\_rent\_pl\_YYYY\_MM.csv

Na potrzeby naszej analizy będziemy korzystać tylko ze zbioru
\textbf{apartments\_pl\_YYYY\_MM.csv}, gdyż będziemy analizować
mieszkania wystawione na sprzedaż, a nie na wynajem.

W datasecie znajduje się 195568 obserwacji.

Dane pochodzą z ogłoszeń, a nie z aktów notarialnych, w związku z czym
mogą zawierać ceny ofertowe, a nie transakcyjne. Ponadto oferty pochodzą
z kolejnych miesięcy, więc część z nich może się powtarzać.

Cały kod jest częścią naszego projektu końcowego z ML, w którym staramy
przewiedzieć ceny mieszkań na podstawie szeregu parametrów, takich jak
miasto, metraż, liczba pokoi, piętro, liczba pięter w budynku, rok
budowy, odległość od centrum miasta, odległość od poszczególnych POI,
oraz udogodnień, takich jak winda, garaż, miejsce parkingowe, ochrona,
balkon i komórka lokatorska.

Ponadto chcielibyśmy znaleźć odpowiedzi na następujących pytania: 1. Jak
rozkładają się ceny mieszkań w poszczególnych miastach. 2. Jakie
parametry mieszkań i ich udogodnienia wpływają na cenę.

    \begin{tcolorbox}[breakable, size=fbox, boxrule=1pt, pad at break*=1mm,colback=cellbackground, colframe=cellborder]
\prompt{In}{incolor}{26}{\boxspacing}
\begin{Verbatim}[commandchars=\\\{\}]
\PY{c+c1}{\PYZsh{} 0. Importowanie bibliotek}
\PY{k+kn}{import}\PY{+w}{ }\PY{n+nn}{pandas}\PY{+w}{ }\PY{k}{as}\PY{+w}{ }\PY{n+nn}{pd}
\PY{k+kn}{import}\PY{+w}{ }\PY{n+nn}{numpy}\PY{+w}{ }\PY{k}{as}\PY{+w}{ }\PY{n+nn}{np}
\PY{k+kn}{import}\PY{+w}{ }\PY{n+nn}{matplotlib}\PY{n+nn}{.}\PY{n+nn}{pyplot}\PY{+w}{ }\PY{k}{as}\PY{+w}{ }\PY{n+nn}{plt}
\PY{k+kn}{import}\PY{+w}{ }\PY{n+nn}{matplotlib}\PY{n+nn}{.}\PY{n+nn}{ticker}\PY{+w}{ }\PY{k}{as}\PY{+w}{ }\PY{n+nn}{ticker}
\PY{k+kn}{import}\PY{+w}{ }\PY{n+nn}{seaborn}\PY{+w}{ }\PY{k}{as}\PY{+w}{ }\PY{n+nn}{sns}
\PY{k+kn}{from}\PY{+w}{ }\PY{n+nn}{pathlib}\PY{+w}{ }\PY{k+kn}{import} \PY{n}{Path}
\PY{k+kn}{import}\PY{+w}{ }\PY{n+nn}{sys}
\PY{k+kn}{import}\PY{+w}{ }\PY{n+nn}{missingno}\PY{+w}{ }\PY{k}{as}\PY{+w}{ }\PY{n+nn}{msno}
\PY{k+kn}{import}\PY{+w}{ }\PY{n+nn}{jinja2}\PY{+w}{ }\PY{k}{as}\PY{+w}{ }\PY{n+nn}{j2}
\PY{k+kn}{from}\PY{+w}{ }\PY{n+nn}{sklearn}\PY{n+nn}{.}\PY{n+nn}{impute}\PY{+w}{ }\PY{k+kn}{import} \PY{n}{KNNImputer}    
\end{Verbatim}
\end{tcolorbox}

    \begin{tcolorbox}[breakable, size=fbox, boxrule=1pt, pad at break*=1mm,colback=cellbackground, colframe=cellborder]
\prompt{In}{incolor}{27}{\boxspacing}
\begin{Verbatim}[commandchars=\\\{\}]
\PY{c+c1}{\PYZsh{} Konfiguracja stylu wykresów}
\PY{n}{plt}\PY{o}{.}\PY{n}{style}\PY{o}{.}\PY{n}{use}\PY{p}{(}\PY{l+s+s1}{\PYZsq{}}\PY{l+s+s1}{seaborn\PYZhy{}v0\PYZus{}8\PYZhy{}whitegrid}\PY{l+s+s1}{\PYZsq{}}\PY{p}{)}
\PY{n}{sns}\PY{o}{.}\PY{n}{set\PYZus{}palette}\PY{p}{(}\PY{l+s+s1}{\PYZsq{}}\PY{l+s+s1}{Set2}\PY{l+s+s1}{\PYZsq{}}\PY{p}{)}
\PY{n}{FIGSIZE} \PY{o}{=} \PY{p}{(}\PY{l+m+mi}{12}\PY{p}{,} \PY{l+m+mi}{8}\PY{p}{)}
\PY{n}{plt}\PY{o}{.}\PY{n}{rcParams}\PY{p}{[}\PY{l+s+s1}{\PYZsq{}}\PY{l+s+s1}{figure.figsize}\PY{l+s+s1}{\PYZsq{}}\PY{p}{]} \PY{o}{=} \PY{n}{FIGSIZE}
\PY{n}{formatter} \PY{o}{=} \PY{n}{ticker}\PY{o}{.}\PY{n}{FuncFormatter}\PY{p}{(}\PY{k}{lambda} \PY{n}{x}\PY{p}{,} \PY{n}{pos}\PY{p}{:} \PY{l+s+sa}{f}\PY{l+s+s1}{\PYZsq{}}\PY{l+s+si}{\PYZob{}}\PY{n}{x}\PY{o}{/}\PY{l+m+mi}{1000}\PY{l+s+si}{:}\PY{l+s+s1}{.0f}\PY{l+s+si}{\PYZcb{}}\PY{l+s+s1}{k}\PY{l+s+s1}{\PYZsq{}}\PY{p}{)}
\end{Verbatim}
\end{tcolorbox}

    Wczytujemy i łączymy wszystkie pliki CSV odnoszące się do sprzedaży
mieszkań.

    \begin{tcolorbox}[breakable, size=fbox, boxrule=1pt, pad at break*=1mm,colback=cellbackground, colframe=cellborder]
\prompt{In}{incolor}{28}{\boxspacing}
\begin{Verbatim}[commandchars=\\\{\}]
\PY{c+c1}{\PYZsh{} Filtrowanie argumentów Jupytera}
\PY{n}{sys}\PY{o}{.}\PY{n}{argv} \PY{o}{=} \PY{p}{[}\PY{n}{arg} \PY{k}{for} \PY{n}{arg} \PY{o+ow}{in} \PY{n}{sys}\PY{o}{.}\PY{n}{argv} \PY{k}{if} \PY{o+ow}{not} \PY{n}{arg}\PY{o}{.}\PY{n}{startswith}\PY{p}{(}\PY{l+s+s1}{\PYZsq{}}\PY{l+s+s1}{\PYZhy{}\PYZhy{}f=}\PY{l+s+s1}{\PYZsq{}}\PY{p}{)}\PY{p}{]}

\PY{c+c1}{\PYZsh{} 1. Wczytywanie danych}
\PY{c+c1}{\PYZsh{} Ustawienie katalogu z danymi}
\PY{c+c1}{\PYZsh{} Można podać katalog jako argument wiersza poleceń lub użyć domyślnego katalogu \PYZsq{}data\PYZsq{}}
\PY{c+c1}{\PYZsh{} Jeśli katalog nie istnieje, zostanie zgłoszony błąd FileNotFoundError}
\PY{k}{if} \PY{n+nb}{len}\PY{p}{(}\PY{n}{sys}\PY{o}{.}\PY{n}{argv}\PY{p}{)} \PY{o}{\PYZgt{}} \PY{l+m+mi}{1}\PY{p}{:}
    \PY{n}{DATA\PYZus{}DIR} \PY{o}{=} \PY{n}{Path}\PY{p}{(}\PY{n}{sys}\PY{o}{.}\PY{n}{argv}\PY{p}{[}\PY{l+m+mi}{1}\PY{p}{]}\PY{p}{)}
\PY{k}{else}\PY{p}{:}
    \PY{n}{DATA\PYZus{}DIR} \PY{o}{=} \PY{n}{Path}\PY{o}{.}\PY{n}{cwd}\PY{p}{(}\PY{p}{)} \PY{o}{/} \PY{l+s+s1}{\PYZsq{}}\PY{l+s+s1}{data}\PY{l+s+s1}{\PYZsq{}}
    
\PY{c+c1}{\PYZsh{} Sprawdzenie czy katalog istnieje}
\PY{k}{if} \PY{o+ow}{not} \PY{n}{DATA\PYZus{}DIR}\PY{o}{.}\PY{n}{exists}\PY{p}{(}\PY{p}{)}\PY{p}{:}
    \PY{k}{raise} \PY{n+ne}{FileNotFoundError}\PY{p}{(}\PY{l+s+sa}{f}\PY{l+s+s2}{\PYZdq{}}\PY{l+s+s2}{Directory does not exist: }\PY{l+s+si}{\PYZob{}}\PY{n}{DATA\PYZus{}DIR}\PY{l+s+si}{\PYZcb{}}\PY{l+s+s2}{\PYZdq{}}\PY{p}{)}
\PY{n+nb}{print}\PY{p}{(}\PY{l+s+sa}{f}\PY{l+s+s2}{\PYZdq{}}\PY{l+s+s2}{Looking for CSV files in: }\PY{l+s+si}{\PYZob{}}\PY{n}{DATA\PYZus{}DIR}\PY{o}{.}\PY{n}{resolve}\PY{p}{(}\PY{p}{)}\PY{l+s+si}{\PYZcb{}}\PY{l+s+s2}{\PYZdq{}}\PY{p}{)}
\PY{n}{csv\PYZus{}files} \PY{o}{=} \PY{p}{[}\PY{n}{f} \PY{k}{for} \PY{n}{f} \PY{o+ow}{in} \PY{n}{DATA\PYZus{}DIR}\PY{o}{.}\PY{n}{glob}\PY{p}{(}\PY{l+s+s1}{\PYZsq{}}\PY{l+s+s1}{apartments\PYZus{}pl\PYZus{}*.csv}\PY{l+s+s1}{\PYZsq{}}\PY{p}{)} \PY{k}{if} \PY{n}{f}\PY{o}{.}\PY{n}{is\PYZus{}file}\PY{p}{(}\PY{p}{)} \PY{o+ow}{and} \PY{n}{f}\PY{o}{.}\PY{n}{suffix}\PY{o}{.}\PY{n}{lower}\PY{p}{(}\PY{p}{)} \PY{o}{==} \PY{l+s+s1}{\PYZsq{}}\PY{l+s+s1}{.csv}\PY{l+s+s1}{\PYZsq{}}\PY{p}{]}

\PY{c+c1}{\PYZsh{} Sprawdzenie czy znaleziono pliki CSV}
\PY{n+nb}{print}\PY{p}{(}\PY{l+s+sa}{f}\PY{l+s+s2}{\PYZdq{}}\PY{l+s+s2}{Found }\PY{l+s+si}{\PYZob{}}\PY{n+nb}{len}\PY{p}{(}\PY{n}{csv\PYZus{}files}\PY{p}{)}\PY{l+s+si}{\PYZcb{}}\PY{l+s+s2}{ CSV files:}\PY{l+s+s2}{\PYZdq{}}\PY{p}{)}
\PY{k}{for} \PY{n}{f} \PY{o+ow}{in} \PY{n}{csv\PYZus{}files}\PY{p}{:}
    \PY{n+nb}{print}\PY{p}{(}\PY{l+s+s2}{\PYZdq{}}\PY{l+s+s2}{ \PYZhy{}}\PY{l+s+s2}{\PYZdq{}}\PY{p}{,} \PY{n}{f}\PY{o}{.}\PY{n}{name}\PY{p}{)}
\PY{k}{if} \PY{n+nb}{len}\PY{p}{(}\PY{n}{csv\PYZus{}files}\PY{p}{)} \PY{o}{==} \PY{l+m+mi}{0}\PY{p}{:}
    \PY{k}{raise} \PY{n+ne}{FileNotFoundError}\PY{p}{(}\PY{l+s+sa}{f}\PY{l+s+s2}{\PYZdq{}}\PY{l+s+s2}{No CSV files found in }\PY{l+s+s2}{\PYZsq{}}\PY{l+s+si}{\PYZob{}}\PY{n}{DATA\PYZus{}DIR}\PY{o}{.}\PY{n}{resolve}\PY{p}{(}\PY{p}{)}\PY{l+s+si}{\PYZcb{}}\PY{l+s+s2}{\PYZsq{}}\PY{l+s+s2}{.}\PY{l+s+s2}{\PYZdq{}}\PY{p}{)}

\PY{c+c1}{\PYZsh{} Wczytywanie plików CSV do DataFrame}
\PY{n}{df\PYZus{}list} \PY{o}{=} \PY{p}{[}\PY{p}{]}
\PY{k}{for} \PY{n}{file} \PY{o+ow}{in} \PY{n}{csv\PYZus{}files}\PY{p}{:}
    \PY{k}{try}\PY{p}{:}
        \PY{n}{df\PYZus{}list}\PY{o}{.}\PY{n}{append}\PY{p}{(}\PY{n}{pd}\PY{o}{.}\PY{n}{read\PYZus{}csv}\PY{p}{(}\PY{n}{file}\PY{p}{)}\PY{p}{)}
    \PY{k}{except} \PY{n+ne}{Exception} \PY{k}{as} \PY{n}{e}\PY{p}{:}
        \PY{n+nb}{print}\PY{p}{(}\PY{l+s+sa}{f}\PY{l+s+s2}{\PYZdq{}}\PY{l+s+s2}{Error loading }\PY{l+s+si}{\PYZob{}}\PY{n}{file}\PY{o}{.}\PY{n}{name}\PY{l+s+si}{\PYZcb{}}\PY{l+s+s2}{: }\PY{l+s+si}{\PYZob{}}\PY{n+nb}{str}\PY{p}{(}\PY{n}{e}\PY{p}{)}\PY{l+s+si}{\PYZcb{}}\PY{l+s+s2}{\PYZdq{}}\PY{p}{)}
\PY{k}{if} \PY{o+ow}{not} \PY{n}{df\PYZus{}list}\PY{p}{:}
    \PY{k}{raise} \PY{n+ne}{RuntimeError}\PY{p}{(}\PY{l+s+s2}{\PYZdq{}}\PY{l+s+s2}{No valid CSV files loaded}\PY{l+s+s2}{\PYZdq{}}\PY{p}{)}

\PY{c+c1}{\PYZsh{} Łączenie wszystkich DataFrame w jeden}
\PY{n}{df} \PY{o}{=} \PY{n}{pd}\PY{o}{.}\PY{n}{concat}\PY{p}{(}\PY{n}{df\PYZus{}list}\PY{p}{,} \PY{n}{ignore\PYZus{}index}\PY{o}{=}\PY{k+kc}{True}\PY{p}{,} \PY{n}{sort}\PY{o}{=}\PY{k+kc}{False}\PY{p}{)}
\PY{n+nb}{print}\PY{p}{(}\PY{l+s+sa}{f}\PY{l+s+s2}{\PYZdq{}}\PY{l+s+s2}{Combined dataframe shape: }\PY{l+s+si}{\PYZob{}}\PY{n}{df}\PY{o}{.}\PY{n}{shape}\PY{l+s+si}{\PYZcb{}}\PY{l+s+s2}{\PYZdq{}}\PY{p}{)}
\end{Verbatim}
\end{tcolorbox}

    \begin{Verbatim}[commandchars=\\\{\}]
Looking for CSV files in: C:\textbackslash{}Users\textbackslash{}olale\textbackslash{}Desktop\textbackslash{}PG\textbackslash{}ML\textbackslash{}ML\_Project\textbackslash{}data
Found 11 CSV files:
 - apartments\_pl\_2023\_08.csv
 - apartments\_pl\_2023\_09.csv
 - apartments\_pl\_2023\_10.csv
 - apartments\_pl\_2023\_11.csv
 - apartments\_pl\_2023\_12.csv
 - apartments\_pl\_2024\_01.csv
 - apartments\_pl\_2024\_02.csv
 - apartments\_pl\_2024\_03.csv
 - apartments\_pl\_2024\_04.csv
 - apartments\_pl\_2024\_05.csv
 - apartments\_pl\_2024\_06.csv
Combined dataframe shape: (195568, 28)
    \end{Verbatim}

    \subsection{Czyszczenie i porządkowanie
danych}\label{czyszczenie-i-porzux105dkowanie-danych}

Zaczynamy od podstawowych działąń na datasecie, typu sprawdzenie danych,
kolumn, statystyk opisowych, unikalnych wartości i zduplikowanych
obserwacji.

    \begin{tcolorbox}[breakable, size=fbox, boxrule=1pt, pad at break*=1mm,colback=cellbackground, colframe=cellborder]
\prompt{In}{incolor}{29}{\boxspacing}
\begin{Verbatim}[commandchars=\\\{\}]
\PY{c+c1}{\PYZsh{} Diagnostyka danych}
\PY{n+nb}{print}\PY{p}{(}\PY{l+s+s2}{\PYZdq{}}\PY{l+s+se}{\PYZbs{}n}\PY{l+s+s2}{\PYZhy{}\PYZhy{}\PYZhy{} Podstawowe informacje o danych \PYZhy{}\PYZhy{}\PYZhy{}}\PY{l+s+s2}{\PYZdq{}}\PY{p}{)}
\PY{n+nb}{print}\PY{p}{(}\PY{n}{df}\PY{o}{.}\PY{n}{head}\PY{p}{(}\PY{p}{)}\PY{p}{)}
\PY{n+nb}{print}\PY{p}{(}\PY{l+s+s2}{\PYZdq{}}\PY{l+s+se}{\PYZbs{}n}\PY{l+s+s2}{\PYZhy{}\PYZhy{}\PYZhy{} Informacje o kolumnach \PYZhy{}\PYZhy{}\PYZhy{}}\PY{l+s+s2}{\PYZdq{}}\PY{p}{)}
\PY{n+nb}{print}\PY{p}{(}\PY{n}{df}\PY{o}{.}\PY{n}{info}\PY{p}{(}\PY{p}{)}\PY{p}{)}
\PY{n+nb}{print}\PY{p}{(}\PY{l+s+s2}{\PYZdq{}}\PY{l+s+se}{\PYZbs{}n}\PY{l+s+s2}{\PYZhy{}\PYZhy{}\PYZhy{} Statystyki \PYZhy{}\PYZhy{}\PYZhy{}}\PY{l+s+s2}{\PYZdq{}}\PY{p}{)}
\PY{n+nb}{print}\PY{p}{(}\PY{n}{df}\PY{o}{.}\PY{n}{describe}\PY{p}{(}\PY{p}{)}\PY{p}{)}
\PY{n+nb}{print}\PY{p}{(}\PY{l+s+s2}{\PYZdq{}}\PY{l+s+se}{\PYZbs{}n}\PY{l+s+s2}{\PYZhy{}\PYZhy{}\PYZhy{} Unikalne wartości \PYZhy{}\PYZhy{}\PYZhy{}}\PY{l+s+s2}{\PYZdq{}}\PY{p}{)}
\PY{n+nb}{print}\PY{p}{(}\PY{n}{df}\PY{o}{.}\PY{n}{nunique}\PY{p}{(}\PY{p}{)}\PY{o}{.}\PY{n}{sort\PYZus{}values}\PY{p}{(}\PY{n}{ascending}\PY{o}{=}\PY{k+kc}{False}\PY{p}{)}\PY{p}{)}

\PY{c+c1}{\PYZsh{} Identyfikacja powtarzających się wartości i ich usunięcie}
\PY{n+nb}{print}\PY{p}{(}\PY{l+s+s2}{\PYZdq{}}\PY{l+s+se}{\PYZbs{}n}\PY{l+s+s2}{\PYZhy{}\PYZhy{}\PYZhy{} Identyfikacja powtarzających się wartości \PYZhy{}\PYZhy{}\PYZhy{}}\PY{l+s+s2}{\PYZdq{}}\PY{p}{)}
\PY{n}{duplicates} \PY{o}{=} \PY{n}{df}\PY{o}{.}\PY{n}{duplicated}\PY{p}{(}\PY{n}{subset}\PY{o}{=}\PY{l+s+s1}{\PYZsq{}}\PY{l+s+s1}{id}\PY{l+s+s1}{\PYZsq{}}\PY{p}{)}\PY{o}{.}\PY{n}{sum}\PY{p}{(}\PY{p}{)}
\PY{n+nb}{print}\PY{p}{(}\PY{l+s+sa}{f}\PY{l+s+s2}{\PYZdq{}}\PY{l+s+s2}{Liczba zduplikowanych wierszy: }\PY{l+s+si}{\PYZob{}}\PY{n}{duplicates}\PY{l+s+si}{\PYZcb{}}\PY{l+s+s2}{\PYZdq{}}\PY{p}{)}
\PY{k}{if} \PY{n}{duplicates} \PY{o}{\PYZgt{}} \PY{l+m+mi}{0}\PY{p}{:}
    \PY{n+nb}{print}\PY{p}{(}\PY{l+s+s2}{\PYZdq{}}\PY{l+s+s2}{Usuwanie zduplikowanych wierszy...}\PY{l+s+s2}{\PYZdq{}}\PY{p}{)}
    \PY{n}{df} \PY{o}{=} \PY{n}{df}\PY{o}{.}\PY{n}{drop\PYZus{}duplicates}\PY{p}{(}\PY{n}{subset}\PY{o}{=}\PY{l+s+s1}{\PYZsq{}}\PY{l+s+s1}{id}\PY{l+s+s1}{\PYZsq{}}\PY{p}{)}
    \PY{n}{df}\PY{o}{.}\PY{n}{duplicated}\PY{p}{(}\PY{n}{subset}\PY{o}{=}\PY{l+s+s1}{\PYZsq{}}\PY{l+s+s1}{id}\PY{l+s+s1}{\PYZsq{}}\PY{p}{)}\PY{o}{.}\PY{n}{sum}\PY{p}{(}\PY{p}{)}
    \PY{n+nb}{print}\PY{p}{(}\PY{l+s+sa}{f}\PY{l+s+s2}{\PYZdq{}}\PY{l+s+s2}{Dane po usunięciu duplikatów: }\PY{l+s+si}{\PYZob{}}\PY{n}{df}\PY{o}{.}\PY{n}{shape}\PY{l+s+si}{\PYZcb{}}\PY{l+s+s2}{\PYZdq{}}\PY{p}{)}


\PY{c+c1}{\PYZsh{} Analiza braków danych}
\PY{n+nb}{print}\PY{p}{(}\PY{l+s+s2}{\PYZdq{}}\PY{l+s+se}{\PYZbs{}n}\PY{l+s+s2}{\PYZhy{}\PYZhy{}\PYZhy{} Analiza brakujących danych \PYZhy{}\PYZhy{}\PYZhy{}}\PY{l+s+s2}{\PYZdq{}}\PY{p}{)}
\PY{n}{missing\PYZus{}percent} \PY{o}{=} \PY{n}{df}\PY{o}{.}\PY{n}{isnull}\PY{p}{(}\PY{p}{)}\PY{o}{.}\PY{n}{mean}\PY{p}{(}\PY{p}{)}\PY{o}{.}\PY{n}{sort\PYZus{}values}\PY{p}{(}\PY{n}{ascending}\PY{o}{=}\PY{k+kc}{False}\PY{p}{)} \PY{o}{*} \PY{l+m+mi}{100}
\PY{n}{missing\PYZus{}percent} \PY{o}{=} \PY{n}{missing\PYZus{}percent}\PY{p}{[}\PY{n}{missing\PYZus{}percent} \PY{o}{\PYZgt{}} \PY{l+m+mi}{0}\PY{p}{]}
\PY{n}{missing\PYZus{}df} \PY{o}{=} \PY{n}{pd}\PY{o}{.}\PY{n}{DataFrame}\PY{p}{(}\PY{p}{\PYZob{}}
    \PY{l+s+s1}{\PYZsq{}}\PY{l+s+s1}{Liczba braków}\PY{l+s+s1}{\PYZsq{}}\PY{p}{:} \PY{n}{df}\PY{o}{.}\PY{n}{isnull}\PY{p}{(}\PY{p}{)}\PY{o}{.}\PY{n}{sum}\PY{p}{(}\PY{p}{)}\PY{p}{[}\PY{n}{missing\PYZus{}percent}\PY{o}{.}\PY{n}{index}\PY{p}{]}\PY{p}{,}
    \PY{l+s+s1}{\PYZsq{}}\PY{l+s+s1}{Rozkład procentowy braków}\PY{l+s+s1}{\PYZsq{}}\PY{p}{:} \PY{n}{missing\PYZus{}percent}\PY{o}{.}\PY{n}{round}\PY{p}{(}\PY{l+m+mi}{2}\PY{p}{)}
\PY{p}{\PYZcb{}}\PY{p}{)}
\PY{n+nb}{print}\PY{p}{(}\PY{n}{missing\PYZus{}df}\PY{p}{)}

\PY{c+c1}{\PYZsh{} Wizualizacja braków danych za pomocą biblioteki missingno}
\PY{n}{plt}\PY{o}{.}\PY{n}{figure}\PY{p}{(}\PY{n}{figsize}\PY{o}{=}\PY{p}{(}\PY{l+m+mi}{12}\PY{p}{,} \PY{l+m+mi}{6}\PY{p}{)}\PY{p}{)}
\PY{n}{msno}\PY{o}{.}\PY{n}{matrix}\PY{p}{(}\PY{n}{df}\PY{p}{)}
\PY{n}{plt}\PY{o}{.}\PY{n}{title}\PY{p}{(}\PY{l+s+s1}{\PYZsq{}}\PY{l+s+s1}{Missing Data Matrix}\PY{l+s+s1}{\PYZsq{}}\PY{p}{)}
\PY{n}{plt}\PY{o}{.}\PY{n}{tight\PYZus{}layout}\PY{p}{(}\PY{p}{)}
\PY{n}{plt}\PY{o}{.}\PY{n}{show}\PY{p}{(}\PY{p}{)}

\PY{c+c1}{\PYZsh{} Wizualizacja braków danych}
\PY{n}{missing\PYZus{}percent\PYZus{}sorted} \PY{o}{=} \PY{p}{(}
    \PY{n}{missing\PYZus{}percent}\PY{p}{[}\PY{n}{missing\PYZus{}percent} \PY{o}{\PYZgt{}} \PY{l+m+mi}{0}\PY{p}{]}
    \PY{o}{.}\PY{n}{sort\PYZus{}values}\PY{p}{(}\PY{n}{ascending}\PY{o}{=}\PY{k+kc}{False}\PY{p}{)}
\PY{p}{)}

\PY{n}{fig}\PY{p}{,} \PY{n}{ax} \PY{o}{=} \PY{n}{plt}\PY{o}{.}\PY{n}{subplots}\PY{p}{(}\PY{n}{figsize}\PY{o}{=}\PY{p}{(}\PY{l+m+mi}{12}\PY{p}{,} \PY{l+m+mi}{6}\PY{p}{)}\PY{p}{)}

\PY{n}{bars} \PY{o}{=} \PY{n}{ax}\PY{o}{.}\PY{n}{barh}\PY{p}{(}
    \PY{n}{missing\PYZus{}percent\PYZus{}sorted}\PY{o}{.}\PY{n}{index}\PY{p}{,}
    \PY{n}{missing\PYZus{}percent\PYZus{}sorted}\PY{o}{.}\PY{n}{values}\PY{p}{,}
    \PY{n}{color}\PY{o}{=}\PY{l+s+s1}{\PYZsq{}}\PY{l+s+s1}{steelblue}\PY{l+s+s1}{\PYZsq{}}
\PY{p}{)}

\PY{n}{ax}\PY{o}{.}\PY{n}{set\PYZus{}title}\PY{p}{(}\PY{l+s+s1}{\PYZsq{}}\PY{l+s+s1}{Procent braków danych w kolumnach}\PY{l+s+s1}{\PYZsq{}}\PY{p}{,} \PY{n}{fontsize}\PY{o}{=}\PY{l+m+mi}{14}\PY{p}{)}
\PY{n}{ax}\PY{o}{.}\PY{n}{set\PYZus{}xlabel}\PY{p}{(}\PY{l+s+s1}{\PYZsq{}}\PY{l+s+s1}{Procent braków (}\PY{l+s+s1}{\PYZpc{}}\PY{l+s+s1}{)}\PY{l+s+s1}{\PYZsq{}}\PY{p}{)}
\PY{n}{ax}\PY{o}{.}\PY{n}{set\PYZus{}ylabel}\PY{p}{(}\PY{l+s+s1}{\PYZsq{}}\PY{l+s+s1}{Kolumny}\PY{l+s+s1}{\PYZsq{}}\PY{p}{)}
\PY{n}{ax}\PY{o}{.}\PY{n}{invert\PYZus{}yaxis}\PY{p}{(}\PY{p}{)}                                     
\PY{n}{ax}\PY{o}{.}\PY{n}{set\PYZus{}xlim}\PY{p}{(}\PY{l+m+mi}{0}\PY{p}{,} \PY{n}{missing\PYZus{}percent\PYZus{}sorted}\PY{o}{.}\PY{n}{max}\PY{p}{(}\PY{p}{)} \PY{o}{*} \PY{l+m+mf}{1.15}\PY{p}{)}

\PY{n}{ax}\PY{o}{.}\PY{n}{bar\PYZus{}label}\PY{p}{(}
    \PY{n}{bars}\PY{p}{,}
    \PY{n}{labels}\PY{o}{=}\PY{p}{[}\PY{l+s+sa}{f}\PY{l+s+s1}{\PYZsq{}}\PY{l+s+si}{\PYZob{}}\PY{n}{v}\PY{l+s+si}{:}\PY{l+s+s1}{.1f}\PY{l+s+si}{\PYZcb{}}\PY{l+s+s1}{\PYZpc{}}\PY{l+s+s1}{\PYZsq{}} \PY{k}{for} \PY{n}{v} \PY{o+ow}{in} \PY{n}{missing\PYZus{}percent\PYZus{}sorted}\PY{o}{.}\PY{n}{values}\PY{p}{]}\PY{p}{,}
    \PY{n}{padding}\PY{o}{=}\PY{l+m+mi}{3}\PY{p}{,}
    \PY{n}{label\PYZus{}type}\PY{o}{=}\PY{l+s+s1}{\PYZsq{}}\PY{l+s+s1}{edge}\PY{l+s+s1}{\PYZsq{}}\PY{p}{,}
    \PY{n}{fontsize}\PY{o}{=}\PY{l+m+mi}{9}\PY{p}{)}

\PY{n}{plt}\PY{o}{.}\PY{n}{tight\PYZus{}layout}\PY{p}{(}\PY{p}{)}
\PY{n}{plt}\PY{o}{.}\PY{n}{show}\PY{p}{(}\PY{p}{)}
\end{Verbatim}
\end{tcolorbox}

    \begin{Verbatim}[commandchars=\\\{\}]

--- Podstawowe informacje o danych ---
                                 id      city          type  squareMeters  \textbackslash{}
0  f8524536d4b09a0c8ccc0197ec9d7bde  szczecin  blockOfFlats         63.00
1  accbe77d4b360fea9735f138a50608dd  szczecin  blockOfFlats         36.00
2  8373aa373dbc3fe7ca3b7434166b8766  szczecin      tenement         73.02
3  0a68cd14c44ec5140143ece75d739535  szczecin      tenement         87.60
4  f66320e153c2441edc0fe293b54c8aeb  szczecin  blockOfFlats         66.00

   rooms  floor  floorCount  buildYear   latitude  longitude  {\ldots}  \textbackslash{}
0    3.0    4.0        10.0     1980.0  53.378933  14.625296  {\ldots}
1    2.0    8.0        10.0        NaN  53.442692  14.559690  {\ldots}
2    3.0    2.0         3.0        NaN  53.452222  14.553333  {\ldots}
3    3.0    2.0         3.0        NaN  53.435100  14.532900  {\ldots}
4    3.0    1.0         3.0        NaN  53.410278  14.503611  {\ldots}

   pharmacyDistance    ownership  buildingMaterial  condition  \textbackslash{}
0             0.413  condominium      concreteSlab        NaN
1             0.205  cooperative      concreteSlab        NaN
2             0.280  condominium             brick        NaN
3             0.087  condominium             brick        NaN
4             0.514  condominium               NaN        NaN

   hasParkingSpace  hasBalcony  hasElevator  hasSecurity  hasStorageRoom  \textbackslash{}
0              yes         yes          yes           no             yes
1               no         yes          yes           no             yes
2               no          no           no           no              no
3              yes         yes           no           no             yes
4               no          no           no           no              no

    price
0  415000
1  395995
2  565000
3  640000
4  759000

[5 rows x 28 columns]

--- Informacje o kolumnach ---
<class 'pandas.core.frame.DataFrame'>
RangeIndex: 195568 entries, 0 to 195567
Data columns (total 28 columns):
 \#   Column                Non-Null Count   Dtype
---  ------                --------------   -----
 0   id                    195568 non-null  object
 1   city                  195568 non-null  object
 2   type                  153307 non-null  object
 3   squareMeters          195568 non-null  float64
 4   rooms                 195568 non-null  float64
 5   floor                 160974 non-null  float64
 6   floorCount            193185 non-null  float64
 7   buildYear             163352 non-null  float64
 8   latitude              195568 non-null  float64
 9   longitude             195568 non-null  float64
 10  centreDistance        195568 non-null  float64
 11  poiCount              195568 non-null  float64
 12  schoolDistance        195400 non-null  float64
 13  clinicDistance        194840 non-null  float64
 14  postOfficeDistance    195320 non-null  float64
 15  kindergartenDistance  195361 non-null  float64
 16  restaurantDistance    195089 non-null  float64
 17  collegeDistance       190132 non-null  float64
 18  pharmacyDistance      195291 non-null  float64
 19  ownership             195568 non-null  object
 20  buildingMaterial      118186 non-null  object
 21  condition             49261 non-null   object
 22  hasParkingSpace       195568 non-null  object
 23  hasBalcony            195568 non-null  object
 24  hasElevator           185866 non-null  object
 25  hasSecurity           195568 non-null  object
 26  hasStorageRoom        195568 non-null  object
 27  price                 195568 non-null  int64
dtypes: float64(16), int64(1), object(11)
memory usage: 41.8+ MB
None

--- Statystyki ---
        squareMeters          rooms          floor     floorCount  \textbackslash{}
count  195568.000000  195568.000000  160974.000000  193185.000000
mean       58.697667       2.679222       3.332414       5.309113
std        21.407206       0.915024       2.531684       3.312234
min        25.000000       1.000000       1.000000       1.000000
25\%        44.000000       2.000000       2.000000       3.000000
50\%        54.600000       3.000000       3.000000       4.000000
75\%        68.550000       3.000000       4.000000       6.000000
max       150.000000       6.000000      29.000000      29.000000

           buildYear       latitude      longitude  centreDistance  \textbackslash{}
count  163352.000000  195568.000000  195568.000000   195568.000000
mean     1985.976346      52.026288      19.465989        4.351114
std        33.812810       1.335275       1.783264        2.835764
min      1850.000000      49.978999      14.447100        0.010000
25\%      1967.000000      51.108796      18.523270        2.010000
50\%      1994.000000      52.194596      19.899434        3.980000
75\%      2016.000000      52.409006      20.989907        6.150000
max      2024.000000      54.606460      23.208873       16.940000

            poiCount  schoolDistance  clinicDistance  postOfficeDistance  \textbackslash{}
count  195568.000000   195400.000000   194840.000000       195320.000000
mean       20.672037        0.412651        0.970287            0.516340
std        24.325708        0.464193        0.888884            0.498013
min         0.000000        0.002000        0.001000            0.001000
25\%         7.000000        0.176000        0.356000            0.239000
50\%        14.000000        0.290000        0.676000            0.393000
75\%        24.000000        0.468000        1.237000            0.623000
max       212.000000        4.946000        4.999000            4.970000

       kindergartenDistance  restaurantDistance  collegeDistance  \textbackslash{}
count         195361.000000       195089.000000     190132.00000
mean               0.367560            0.345257          1.44327
std                0.444673            0.463510          1.10457
min                0.001000            0.001000          0.00400
25\%                0.156000            0.114000          0.57700
50\%                0.262000            0.229000          1.12000
75\%                0.416000            0.409000          2.05500
max                4.961000            4.985000          5.00000

       pharmacyDistance         price
count     195291.000000  1.955680e+05
mean           0.358114  7.841833e+05
std            0.457679  4.097092e+05
min            0.001000  1.500000e+05
25\%            0.142000  5.200000e+05
50\%            0.239000  6.990000e+05
75\%            0.406000  9.300000e+05
max            4.992000  3.250000e+06

--- Unikalne wartości ---
id                      92967
longitude               49315
latitude                47484
price                    8092
squareMeters             7201
collegeDistance          4825
clinicDistance           4282
postOfficeDistance       2713
schoolDistance           2537
pharmacyDistance         2446
restaurantDistance       2437
kindergartenDistance     2342
centreDistance           1489
poiCount                  196
buildYear                 166
floorCount                 29
floor                      27
city                       15
rooms                       6
type                        3
ownership                   3
buildingMaterial            2
condition                   2
hasParkingSpace             2
hasBalcony                  2
hasElevator                 2
hasSecurity                 2
hasStorageRoom              2
dtype: int64

--- Identyfikacja powtarzających się wartości ---
Liczba zduplikowanych wierszy: 102601
Usuwanie zduplikowanych wierszy{\ldots}
Dane po usunięciu duplikatów: (92967, 28)

--- Analiza brakujących danych ---
                      Liczba braków  Rozkład procentowy braków
condition                     69914                      75.20
buildingMaterial              39179                      42.14
type                          19792                      21.29
floor                         15984                      17.19
buildYear                     15641                      16.82
hasElevator                    4448                       4.78
collegeDistance                2492                       2.68
floorCount                     1082                       1.16
clinicDistance                  333                       0.36
restaurantDistance              226                       0.24
pharmacyDistance                128                       0.14
postOfficeDistance               99                       0.11
kindergartenDistance             85                       0.09
schoolDistance                   60                       0.06
    \end{Verbatim}

    \begin{Verbatim}[commandchars=\\\{\}]
C:\textbackslash{}Users\textbackslash{}olale\textbackslash{}AppData\textbackslash{}Local\textbackslash{}Temp\textbackslash{}ipykernel\_12048\textbackslash{}731635470.py:36: UserWarning:
This figure includes Axes that are not compatible with tight\_layout, so results
might be incorrect.
  plt.tight\_layout()
    \end{Verbatim}

    
    \begin{Verbatim}[commandchars=\\\{\}]
<Figure size 1200x600 with 0 Axes>
    \end{Verbatim}

    
    \begin{center}
    \adjustimage{max size={0.9\linewidth}{0.9\paperheight}}{Analiza_mieszkan_Final_files/Analiza_mieszkan_Final_6_3.png}
    \end{center}
    { \hspace*{\fill} \\}
    
    \begin{center}
    \adjustimage{max size={0.9\linewidth}{0.9\paperheight}}{Analiza_mieszkan_Final_files/Analiza_mieszkan_Final_6_4.png}
    \end{center}
    { \hspace*{\fill} \\}
    
    Połączony zbiór danych zawiera ok. 196 tys. wierszy, a po usunięciu
duplikatów zostaje niecałe 93 tys. wierszy (spadek o 52\%). Jest to
działanie zamierzone, gdyż interesuje nas analiza danych dotyczących
mieszkań, a nie samych ogłoszeń, więc potrzebujemy tylko jednej
obserwacji na mieszkanie.

Jak widać na powyższych wizualizacjach, brakuje sporo danych, zwłaszcza
w kolumnach `condition' (prawie 75\%) oraz `buildingMaterial' (ponad
42\%). Rozważamy ich usunięcie, a w przypadku kolumn z mniejszą liczbą
braków - wypełnienie metodą KNN-imputacji osobno dla każdego miasta
(KNNImputer, k=5), co pozwoli zachować większą liczbę pełnych obserwacji
niż przy prostym uzupełnianiu medianą dla całej populacji.

    \begin{tcolorbox}[breakable, size=fbox, boxrule=1pt, pad at break*=1mm,colback=cellbackground, colframe=cellborder]
\prompt{In}{incolor}{ }{\boxspacing}
\begin{Verbatim}[commandchars=\\\{\}]
\PY{c+c1}{\PYZsh{} Strategia imputacji i obsługa braków}
\PY{c+c1}{\PYZsh{} Usunięcie kolumn z \PYZgt{}75\PYZpc{} braków}
\PY{n}{cols\PYZus{}to\PYZus{}drop} \PY{o}{=} \PY{p}{[}\PY{l+s+s1}{\PYZsq{}}\PY{l+s+s1}{condition}\PY{l+s+s1}{\PYZsq{}}\PY{p}{]}
\PY{n}{df} \PY{o}{=} \PY{n}{df}\PY{o}{.}\PY{n}{drop}\PY{p}{(}\PY{n}{columns}\PY{o}{=}\PY{n}{cols\PYZus{}to\PYZus{}drop}\PY{p}{,} \PY{n}{errors}\PY{o}{=}\PY{l+s+s1}{\PYZsq{}}\PY{l+s+s1}{ignore}\PY{l+s+s1}{\PYZsq{}}\PY{p}{)}

\PY{c+c1}{\PYZsh{} 1. KNNImputer do braków liczbowych}
\PY{k}{def}\PY{+w}{ }\PY{n+nf}{knn\PYZus{}impute\PYZus{}by\PYZus{}city}\PY{p}{(}\PY{n}{df}\PY{p}{,} \PY{n}{numeric\PYZus{}cols}\PY{p}{)}\PY{p}{:}
    \PY{n}{frames} \PY{o}{=} \PY{p}{[}\PY{p}{]}
    \PY{k}{for} \PY{n}{city}\PY{p}{,} \PY{n}{group} \PY{o+ow}{in} \PY{n}{df}\PY{o}{.}\PY{n}{groupby}\PY{p}{(}\PY{l+s+s1}{\PYZsq{}}\PY{l+s+s1}{city}\PY{l+s+s1}{\PYZsq{}}\PY{p}{)}\PY{p}{:}
        \PY{n}{imputer} \PY{o}{=} \PY{n}{KNNImputer}\PY{p}{(}\PY{n}{n\PYZus{}neighbors}\PY{o}{=}\PY{l+m+mi}{5}\PY{p}{)}
        \PY{n}{group\PYZus{}num} \PY{o}{=} \PY{n}{group}\PY{p}{[}\PY{n}{numeric\PYZus{}cols}\PY{p}{]}\PY{o}{.}\PY{n}{copy}\PY{p}{(}\PY{p}{)}
        \PY{n}{imputed} \PY{o}{=} \PY{n}{imputer}\PY{o}{.}\PY{n}{fit\PYZus{}transform}\PY{p}{(}\PY{n}{group\PYZus{}num}\PY{p}{)}
        \PY{n}{group}\PY{o}{.}\PY{n}{loc}\PY{p}{[}\PY{p}{:}\PY{p}{,} \PY{n}{numeric\PYZus{}cols}\PY{p}{]} \PY{o}{=} \PY{n}{imputed}
        \PY{n}{frames}\PY{o}{.}\PY{n}{append}\PY{p}{(}\PY{n}{group}\PY{p}{)}
    \PY{k}{return} \PY{n}{pd}\PY{o}{.}\PY{n}{concat}\PY{p}{(}\PY{n}{frames}\PY{p}{)}

\PY{n}{numeric\PYZus{}cols} \PY{o}{=} \PY{p}{[}
    \PY{l+s+s1}{\PYZsq{}}\PY{l+s+s1}{floor}\PY{l+s+s1}{\PYZsq{}}\PY{p}{,} \PY{l+s+s1}{\PYZsq{}}\PY{l+s+s1}{floorCount}\PY{l+s+s1}{\PYZsq{}}\PY{p}{,} \PY{l+s+s1}{\PYZsq{}}\PY{l+s+s1}{buildYear}\PY{l+s+s1}{\PYZsq{}}\PY{p}{,}
    \PY{l+s+s1}{\PYZsq{}}\PY{l+s+s1}{schoolDistance}\PY{l+s+s1}{\PYZsq{}}\PY{p}{,} \PY{l+s+s1}{\PYZsq{}}\PY{l+s+s1}{clinicDistance}\PY{l+s+s1}{\PYZsq{}}\PY{p}{,} \PY{l+s+s1}{\PYZsq{}}\PY{l+s+s1}{postOfficeDistance}\PY{l+s+s1}{\PYZsq{}}\PY{p}{,}
    \PY{l+s+s1}{\PYZsq{}}\PY{l+s+s1}{kindergartenDistance}\PY{l+s+s1}{\PYZsq{}}\PY{p}{,} \PY{l+s+s1}{\PYZsq{}}\PY{l+s+s1}{restaurantDistance}\PY{l+s+s1}{\PYZsq{}}\PY{p}{,} \PY{l+s+s1}{\PYZsq{}}\PY{l+s+s1}{collegeDistance}\PY{l+s+s1}{\PYZsq{}}\PY{p}{,}
    \PY{l+s+s1}{\PYZsq{}}\PY{l+s+s1}{pharmacyDistance}\PY{l+s+s1}{\PYZsq{}}
\PY{p}{]}

\PY{c+c1}{\PYZsh{} Zamiana kolumn na liczby}
\PY{k}{for} \PY{n}{col} \PY{o+ow}{in} \PY{n}{numeric\PYZus{}cols}\PY{p}{:}
    \PY{k}{if} \PY{n}{col} \PY{o+ow}{in} \PY{n}{df}\PY{o}{.}\PY{n}{columns}\PY{p}{:}
        \PY{n}{df}\PY{p}{[}\PY{n}{col}\PY{p}{]} \PY{o}{=} \PY{n}{pd}\PY{o}{.}\PY{n}{to\PYZus{}numeric}\PY{p}{(}\PY{n}{df}\PY{p}{[}\PY{n}{col}\PY{p}{]}\PY{p}{,} \PY{n}{errors}\PY{o}{=}\PY{l+s+s1}{\PYZsq{}}\PY{l+s+s1}{coerce}\PY{l+s+s1}{\PYZsq{}}\PY{p}{)}

\PY{n+nb}{print}\PY{p}{(}\PY{l+s+s2}{\PYZdq{}}\PY{l+s+s2}{Imputacja braków za pomocą KNNImputer}\PY{l+s+s2}{\PYZdq{}}\PY{p}{)}
\PY{n}{df} \PY{o}{=} \PY{n}{knn\PYZus{}impute\PYZus{}by\PYZus{}city}\PY{p}{(}\PY{n}{df}\PY{p}{,} \PY{n}{numeric\PYZus{}cols}\PY{p}{)}

\PY{c+c1}{\PYZsh{} 2. Imputacja modą dla zmiennych kategorycznych}
\PY{n}{cat\PYZus{}cols} \PY{o}{=} \PY{p}{[}\PY{l+s+s1}{\PYZsq{}}\PY{l+s+s1}{type}\PY{l+s+s1}{\PYZsq{}}\PY{p}{,} \PY{l+s+s1}{\PYZsq{}}\PY{l+s+s1}{buildingMaterial}\PY{l+s+s1}{\PYZsq{}}\PY{p}{,} \PY{l+s+s1}{\PYZsq{}}\PY{l+s+s1}{hasElevator}\PY{l+s+s1}{\PYZsq{}}\PY{p}{]}
\PY{k}{for} \PY{n}{col} \PY{o+ow}{in} \PY{n}{cat\PYZus{}cols}\PY{p}{:}
    \PY{k}{if} \PY{n}{col} \PY{o+ow}{in} \PY{n}{df}\PY{o}{.}\PY{n}{columns}\PY{p}{:}
        \PY{n}{mode\PYZus{}val} \PY{o}{=} \PY{n}{df}\PY{p}{[}\PY{n}{col}\PY{p}{]}\PY{o}{.}\PY{n}{mode}\PY{p}{(}\PY{n}{dropna}\PY{o}{=}\PY{k+kc}{True}\PY{p}{)}\PY{p}{[}\PY{l+m+mi}{0}\PY{p}{]}      \PY{c+c1}{\PYZsh{} dominująca wartość}
        \PY{n}{df}\PY{p}{[}\PY{n}{col}\PY{p}{]} \PY{o}{=} \PY{n}{df}\PY{p}{[}\PY{n}{col}\PY{p}{]}\PY{o}{.}\PY{n}{fillna}\PY{p}{(}\PY{n}{mode\PYZus{}val}\PY{p}{)}           \PY{c+c1}{\PYZsh{} brak chained‑assignment}

\PY{c+c1}{\PYZsh{} 3. Usunięcie pozostałych braków}
\PY{n}{before\PYZus{}drop} \PY{o}{=} \PY{n+nb}{len}\PY{p}{(}\PY{n}{df}\PY{p}{)}
\PY{n}{df} \PY{o}{=} \PY{n}{df}\PY{o}{.}\PY{n}{dropna}\PY{p}{(}\PY{p}{)}
\PY{n}{after\PYZus{}drop}  \PY{o}{=} \PY{n+nb}{len}\PY{p}{(}\PY{n}{df}\PY{p}{)}
\PY{n+nb}{print}\PY{p}{(}\PY{l+s+sa}{f}\PY{l+s+s2}{\PYZdq{}}\PY{l+s+s2}{Usunięto }\PY{l+s+si}{\PYZob{}}\PY{n}{before\PYZus{}drop}\PY{+w}{ }\PY{o}{\PYZhy{}}\PY{+w}{ }\PY{n}{after\PYZus{}drop}\PY{l+s+si}{\PYZcb{}}\PY{l+s+s2}{ wierszy }\PY{l+s+s2}{\PYZdq{}}
      \PY{l+s+sa}{f}\PY{l+s+s2}{\PYZdq{}}\PY{l+s+s2}{(}\PY{l+s+si}{\PYZob{}}\PY{p}{(}\PY{n}{before\PYZus{}drop}\PY{+w}{ }\PY{o}{\PYZhy{}}\PY{+w}{ }\PY{n}{after\PYZus{}drop}\PY{p}{)}\PY{+w}{ }\PY{o}{/}\PY{+w}{ }\PY{n}{before\PYZus{}drop}\PY{l+s+si}{:}\PY{l+s+s2}{.2\PYZpc{}}\PY{l+s+si}{\PYZcb{}}\PY{l+s+s2}{)}\PY{l+s+s2}{\PYZdq{}}\PY{p}{)}
\end{Verbatim}
\end{tcolorbox}

    \begin{Verbatim}[commandchars=\\\{\}]
Imputuję braki za pomocą KNNImputer
Usunięto 0 wierszy (0.00\%)
    \end{Verbatim}

    \begin{tcolorbox}[breakable, size=fbox, boxrule=1pt, pad at break*=1mm,colback=cellbackground, colframe=cellborder]
\prompt{In}{incolor}{ }{\boxspacing}
\begin{Verbatim}[commandchars=\\\{\}]
\PY{c+c1}{\PYZsh{} Sprawdzenie, czy zostały jakieś braki}
\PY{n}{missing\PYZus{}after} \PY{o}{=} \PY{n}{df}\PY{o}{.}\PY{n}{isnull}\PY{p}{(}\PY{p}{)}\PY{o}{.}\PY{n}{sum}\PY{p}{(}\PY{p}{)}\PY{o}{.}\PY{n}{sum}\PY{p}{(}\PY{p}{)}
\PY{k}{if} \PY{n}{missing\PYZus{}after} \PY{o}{\PYZgt{}} \PY{l+m+mi}{0}\PY{p}{:}
    \PY{n+nb}{print}\PY{p}{(}\PY{l+s+sa}{f}\PY{l+s+s2}{\PYZdq{}}\PY{l+s+s2}{Uwaga: W datasecie wciąż brakuje }\PY{l+s+si}{\PYZob{}}\PY{n}{missing\PYZus{}after}\PY{l+s+si}{\PYZcb{}}\PY{l+s+s2}{ wartości.}\PY{l+s+s2}{\PYZdq{}}\PY{p}{)}
\PY{c+c1}{\PYZsh{} Wizualizacja braków danych za pomocą biblioteki missingno po imputacji}
\PY{n}{plt}\PY{o}{.}\PY{n}{figure}\PY{p}{(}\PY{n}{figsize}\PY{o}{=}\PY{p}{(}\PY{l+m+mi}{12}\PY{p}{,} \PY{l+m+mi}{6}\PY{p}{)}\PY{p}{)}
\PY{n}{msno}\PY{o}{.}\PY{n}{matrix}\PY{p}{(}\PY{n}{df}\PY{p}{)}
\PY{n}{plt}\PY{o}{.}\PY{n}{title}\PY{p}{(}\PY{l+s+s1}{\PYZsq{}}\PY{l+s+s1}{Wykres braków danych po imputacji}\PY{l+s+s1}{\PYZsq{}}\PY{p}{)}
\PY{n}{plt}\PY{o}{.}\PY{n}{tight\PYZus{}layout}\PY{p}{(}\PY{p}{)}
\PY{n}{plt}\PY{o}{.}\PY{n}{show}\PY{p}{(}\PY{p}{)}
\end{Verbatim}
\end{tcolorbox}

    \begin{Verbatim}[commandchars=\\\{\}]
C:\textbackslash{}Users\textbackslash{}olale\textbackslash{}AppData\textbackslash{}Local\textbackslash{}Temp\textbackslash{}ipykernel\_12048\textbackslash{}2023313051.py:9: UserWarning:
This figure includes Axes that are not compatible with tight\_layout, so results
might be incorrect.
  plt.tight\_layout()
    \end{Verbatim}

    
    \begin{Verbatim}[commandchars=\\\{\}]
<Figure size 1200x600 with 0 Axes>
    \end{Verbatim}

    
    \begin{center}
    \adjustimage{max size={0.9\linewidth}{0.9\paperheight}}{Analiza_mieszkan_Final_files/Analiza_mieszkan_Final_9_2.png}
    \end{center}
    { \hspace*{\fill} \\}
    
    Ceny mieszkań wykazują dużą zmienność i często posiadają wartości
odstające (tzw. outliery), które mogą zaburzać wyniki analiz. W celu
lepszego zrozumienia rozkładu cen, wykonaliśmy histogram oraz wykres
typu boxplot przedstawiający pełen zakres zmienności.

    \begin{tcolorbox}[breakable, size=fbox, boxrule=1pt, pad at break*=1mm,colback=cellbackground, colframe=cellborder]
\prompt{In}{incolor}{32}{\boxspacing}
\begin{Verbatim}[commandchars=\\\{\}]
\PY{c+c1}{\PYZsh{} Obserwacje odstające}

\PY{c+c1}{\PYZsh{} Histogram cen mieszkań}
\PY{n}{fig}\PY{p}{,} \PY{n}{ax} \PY{o}{=} \PY{n}{plt}\PY{o}{.}\PY{n}{subplots}\PY{p}{(}\PY{l+m+mi}{1}\PY{p}{,} \PY{l+m+mi}{1}\PY{p}{,} \PY{n}{figsize}\PY{o}{=}\PY{p}{(}\PY{l+m+mi}{12}\PY{p}{,} \PY{l+m+mi}{6}\PY{p}{)}\PY{p}{)}
\PY{n}{fmt} \PY{o}{=} \PY{n}{ticker}\PY{o}{.}\PY{n}{FuncFormatter}\PY{p}{(}\PY{k}{lambda} \PY{n}{x}\PY{p}{,} \PY{n}{pos}\PY{p}{:} \PY{l+s+sa}{f}\PY{l+s+s1}{\PYZsq{}}\PY{l+s+si}{\PYZob{}}\PY{n}{x}\PY{o}{/}\PY{l+m+mi}{1000}\PY{l+s+si}{:}\PY{l+s+s1}{.0f}\PY{l+s+si}{\PYZcb{}}\PY{l+s+s1}{ tys.}\PY{l+s+s1}{\PYZsq{}}\PY{p}{)}
\PY{n}{sns}\PY{o}{.}\PY{n}{histplot}\PY{p}{(}\PY{n}{df}\PY{p}{[}\PY{l+s+s1}{\PYZsq{}}\PY{l+s+s1}{price}\PY{l+s+s1}{\PYZsq{}}\PY{p}{]}\PY{p}{,} \PY{n}{bins}\PY{o}{=}\PY{l+m+mi}{50}\PY{p}{,} \PY{n}{kde}\PY{o}{=}\PY{k+kc}{True}\PY{p}{,} \PY{n}{ax}\PY{o}{=}\PY{n}{ax}\PY{p}{,} \PY{n}{color}\PY{o}{=}\PY{l+s+s1}{\PYZsq{}}\PY{l+s+s1}{skyblue}\PY{l+s+s1}{\PYZsq{}}\PY{p}{)}
\PY{n}{ax}\PY{o}{.}\PY{n}{set\PYZus{}title}\PY{p}{(}\PY{l+s+s1}{\PYZsq{}}\PY{l+s+s1}{Histogram cen mieszkań}\PY{l+s+s1}{\PYZsq{}}\PY{p}{)}
\PY{n}{ax}\PY{o}{.}\PY{n}{set\PYZus{}xlabel}\PY{p}{(}\PY{l+s+s1}{\PYZsq{}}\PY{l+s+s1}{Cena (PLN)}\PY{l+s+s1}{\PYZsq{}}\PY{p}{)}
\PY{n}{ax}\PY{o}{.}\PY{n}{set\PYZus{}ylabel}\PY{p}{(}\PY{l+s+s1}{\PYZsq{}}\PY{l+s+s1}{Liczba mieszkań}\PY{l+s+s1}{\PYZsq{}}\PY{p}{)}
\PY{n}{ax}\PY{o}{.}\PY{n}{grid}\PY{p}{(}\PY{k+kc}{True}\PY{p}{,} \PY{n}{linestyle}\PY{o}{=}\PY{l+s+s1}{\PYZsq{}}\PY{l+s+s1}{\PYZhy{}\PYZhy{}}\PY{l+s+s1}{\PYZsq{}}\PY{p}{,} \PY{n}{alpha}\PY{o}{=}\PY{l+m+mf}{0.5}\PY{p}{)}
\PY{n}{ax}\PY{o}{.}\PY{n}{xaxis}\PY{o}{.}\PY{n}{set\PYZus{}major\PYZus{}formatter}\PY{p}{(}\PY{n}{fmt}\PY{p}{)}
\PY{n}{plt}\PY{o}{.}\PY{n}{show}\PY{p}{(}\PY{p}{)}

\PY{c+c1}{\PYZsh{} Boxplot ceny mieszkań}
\PY{n}{fig}\PY{p}{,} \PY{n}{ax} \PY{o}{=} \PY{n}{plt}\PY{o}{.}\PY{n}{subplots}\PY{p}{(}\PY{l+m+mi}{1}\PY{p}{,} \PY{l+m+mi}{1}\PY{p}{,} \PY{n}{figsize}\PY{o}{=}\PY{p}{(}\PY{l+m+mi}{12}\PY{p}{,} \PY{l+m+mi}{6}\PY{p}{)}\PY{p}{)}
\PY{n}{fmt} \PY{o}{=} \PY{n}{ticker}\PY{o}{.}\PY{n}{FuncFormatter}\PY{p}{(}\PY{k}{lambda} \PY{n}{x}\PY{p}{,} \PY{n}{pos}\PY{p}{:} \PY{l+s+sa}{f}\PY{l+s+s1}{\PYZsq{}}\PY{l+s+si}{\PYZob{}}\PY{n}{x}\PY{o}{/}\PY{l+m+mi}{1000}\PY{l+s+si}{:}\PY{l+s+s1}{.0f}\PY{l+s+si}{\PYZcb{}}\PY{l+s+s1}{ tys.}\PY{l+s+s1}{\PYZsq{}}\PY{p}{)}
\PY{n}{sns}\PY{o}{.}\PY{n}{boxplot}\PY{p}{(}\PY{n}{x}\PY{o}{=}\PY{n}{df}\PY{p}{[}\PY{l+s+s1}{\PYZsq{}}\PY{l+s+s1}{price}\PY{l+s+s1}{\PYZsq{}}\PY{p}{]}\PY{p}{,} \PY{n}{ax}\PY{o}{=}\PY{n}{ax}\PY{p}{,} \PY{n}{color}\PY{o}{=}\PY{l+s+s1}{\PYZsq{}}\PY{l+s+s1}{skyblue}\PY{l+s+s1}{\PYZsq{}}\PY{p}{)}
\PY{n}{ax}\PY{o}{.}\PY{n}{set\PYZus{}title}\PY{p}{(}\PY{l+s+s1}{\PYZsq{}}\PY{l+s+s1}{Rozkład cen mieszkań}\PY{l+s+s1}{\PYZsq{}}\PY{p}{)}
\PY{n}{ax}\PY{o}{.}\PY{n}{set\PYZus{}xlabel}\PY{p}{(}\PY{l+s+s1}{\PYZsq{}}\PY{l+s+s1}{Cena (PLN)}\PY{l+s+s1}{\PYZsq{}}\PY{p}{)}
\PY{n}{ax}\PY{o}{.}\PY{n}{xaxis}\PY{o}{.}\PY{n}{set\PYZus{}major\PYZus{}formatter}\PY{p}{(}\PY{n}{fmt}\PY{p}{)}
\PY{n}{plt}\PY{o}{.}\PY{n}{show}\PY{p}{(}\PY{p}{)}
\end{Verbatim}
\end{tcolorbox}

    \begin{center}
    \adjustimage{max size={0.9\linewidth}{0.9\paperheight}}{Analiza_mieszkan_Final_files/Analiza_mieszkan_Final_11_0.png}
    \end{center}
    { \hspace*{\fill} \\}
    
    \begin{center}
    \adjustimage{max size={0.9\linewidth}{0.9\paperheight}}{Analiza_mieszkan_Final_files/Analiza_mieszkan_Final_11_1.png}
    \end{center}
    { \hspace*{\fill} \\}
    
    Na boxplocie możemy dostrzec długi ogon prawostronny. Ze względu na
obecność wartości skrajnie wysokich lub niskich, ograniczono dane do
zakresu między 1. a 95. percentylem rozkładu cen. Takie podejście
pozwala zachować 94\% głównych obserwacji, eliminując przypadki
odstające, które mogłyby zniekształcić wnioski statystyczne i
wizualizacje.

Dla poprawy symetrii rozkładu danych i ewentualnych analiz
statystycznych można rozważyć transformację zmiennej `price', np.
poprzez zastosowanie logarytmu.

    \begin{tcolorbox}[breakable, size=fbox, boxrule=1pt, pad at break*=1mm,colback=cellbackground, colframe=cellborder]
\prompt{In}{incolor}{33}{\boxspacing}
\begin{Verbatim}[commandchars=\\\{\}]
\PY{c+c1}{\PYZsh{} Sprawdzenie liczby obserwacji przed filtrowaniem}
\PY{n+nb}{print}\PY{p}{(}\PY{l+s+sa}{f}\PY{l+s+s2}{\PYZdq{}}\PY{l+s+s2}{Liczba obserwacji: }\PY{l+s+si}{\PYZob{}}\PY{n}{df}\PY{o}{.}\PY{n}{shape}\PY{l+s+si}{\PYZcb{}}\PY{l+s+s2}{\PYZdq{}}\PY{p}{)}

\PY{c+c1}{\PYZsh{} Ustawienia percentyli}
\PY{n}{lower\PYZus{}p} \PY{o}{=} \PY{l+m+mf}{0.01}
\PY{n}{upper\PYZus{}p} \PY{o}{=} \PY{l+m+mf}{0.95}

\PY{c+c1}{\PYZsh{} Przefiltrowanie outlierów osobno dla każdego miasta}
\PY{n}{filtered\PYZus{}df\PYZus{}list} \PY{o}{=} \PY{p}{[}\PY{p}{]}

\PY{k}{for} \PY{n}{city}\PY{p}{,} \PY{n}{group} \PY{o+ow}{in} \PY{n}{df}\PY{o}{.}\PY{n}{groupby}\PY{p}{(}\PY{l+s+s1}{\PYZsq{}}\PY{l+s+s1}{city}\PY{l+s+s1}{\PYZsq{}}\PY{p}{)}\PY{p}{:}
    \PY{n}{p\PYZus{}low} \PY{o}{=} \PY{n}{group}\PY{p}{[}\PY{l+s+s1}{\PYZsq{}}\PY{l+s+s1}{price}\PY{l+s+s1}{\PYZsq{}}\PY{p}{]}\PY{o}{.}\PY{n}{quantile}\PY{p}{(}\PY{n}{lower\PYZus{}p}\PY{p}{)}
    \PY{n}{p\PYZus{}high} \PY{o}{=} \PY{n}{group}\PY{p}{[}\PY{l+s+s1}{\PYZsq{}}\PY{l+s+s1}{price}\PY{l+s+s1}{\PYZsq{}}\PY{p}{]}\PY{o}{.}\PY{n}{quantile}\PY{p}{(}\PY{n}{upper\PYZus{}p}\PY{p}{)}
    \PY{n}{filtered} \PY{o}{=} \PY{n}{group}\PY{p}{[}\PY{p}{(}\PY{n}{group}\PY{p}{[}\PY{l+s+s1}{\PYZsq{}}\PY{l+s+s1}{price}\PY{l+s+s1}{\PYZsq{}}\PY{p}{]} \PY{o}{\PYZgt{}}\PY{o}{=} \PY{n}{p\PYZus{}low}\PY{p}{)} \PY{o}{\PYZam{}} \PY{p}{(}\PY{n}{group}\PY{p}{[}\PY{l+s+s1}{\PYZsq{}}\PY{l+s+s1}{price}\PY{l+s+s1}{\PYZsq{}}\PY{p}{]} \PY{o}{\PYZlt{}}\PY{o}{=} \PY{n}{p\PYZus{}high}\PY{p}{)}\PY{p}{]}
    \PY{n}{filtered\PYZus{}df\PYZus{}list}\PY{o}{.}\PY{n}{append}\PY{p}{(}\PY{n}{filtered}\PY{p}{)}
    \PY{n+nb}{print}\PY{p}{(}\PY{l+s+sa}{f}\PY{l+s+s2}{\PYZdq{}}\PY{l+s+si}{\PYZob{}}\PY{n}{city}\PY{l+s+si}{\PYZcb{}}\PY{l+s+s2}{: }\PY{l+s+si}{\PYZob{}}\PY{n}{group}\PY{o}{.}\PY{n}{shape}\PY{p}{[}\PY{l+m+mi}{0}\PY{p}{]}\PY{l+s+si}{\PYZcb{}}\PY{l+s+s2}{ → }\PY{l+s+si}{\PYZob{}}\PY{n}{filtered}\PY{o}{.}\PY{n}{shape}\PY{p}{[}\PY{l+m+mi}{0}\PY{p}{]}\PY{l+s+si}{\PYZcb{}}\PY{l+s+s2}{ (usunięto }\PY{l+s+si}{\PYZob{}}\PY{n}{group}\PY{o}{.}\PY{n}{shape}\PY{p}{[}\PY{l+m+mi}{0}\PY{p}{]}\PY{+w}{ }\PY{o}{\PYZhy{}}\PY{+w}{ }\PY{n}{filtered}\PY{o}{.}\PY{n}{shape}\PY{p}{[}\PY{l+m+mi}{0}\PY{p}{]}\PY{l+s+si}{\PYZcb{}}\PY{l+s+s2}{)}\PY{l+s+s2}{\PYZdq{}}\PY{p}{)}

\PY{c+c1}{\PYZsh{} Połączenie danych po odfiltrowaniu}
\PY{n}{df} \PY{o}{=} \PY{n}{pd}\PY{o}{.}\PY{n}{concat}\PY{p}{(}\PY{n}{filtered\PYZus{}df\PYZus{}list}\PY{p}{)}
\PY{n+nb}{print}\PY{p}{(}\PY{l+s+sa}{f}\PY{l+s+s2}{\PYZdq{}}\PY{l+s+s2}{Liczba obserwacji po usunięciu wartości odstających: }\PY{l+s+si}{\PYZob{}}\PY{n}{df}\PY{o}{.}\PY{n}{shape}\PY{l+s+si}{\PYZcb{}}\PY{l+s+s2}{\PYZdq{}}\PY{p}{)}
\end{Verbatim}
\end{tcolorbox}

    \begin{Verbatim}[commandchars=\\\{\}]
Liczba obserwacji: (92967, 27)
bialystok: 1017 → 955 (usunięto 62)
bydgoszcz: 3627 → 3414 (usunięto 213)
czestochowa: 1242 → 1171 (usunięto 71)
gdansk: 8244 → 7749 (usunięto 495)
gdynia: 3569 → 3354 (usunięto 215)
katowice: 2397 → 2258 (usunięto 139)
krakow: 14186 → 13337 (usunięto 849)
lodz: 7073 → 6657 (usunięto 416)
lublin: 2299 → 2162 (usunięto 137)
poznan: 3639 → 3431 (usunięto 208)
radom: 845 → 794 (usunięto 51)
rzeszow: 762 → 715 (usunięto 47)
szczecin: 2487 → 2340 (usunięto 147)
warszawa: 31982 → 30076 (usunięto 1906)
wroclaw: 9598 → 9030 (usunięto 568)
Liczba obserwacji po usunięciu wartości odstających: (87443, 27)
    \end{Verbatim}

    \begin{tcolorbox}[breakable, size=fbox, boxrule=1pt, pad at break*=1mm,colback=cellbackground, colframe=cellborder]
\prompt{In}{incolor}{ }{\boxspacing}
\begin{Verbatim}[commandchars=\\\{\}]
\PY{c+c1}{\PYZsh{} Histogram cen po oczyszczeniu}
\PY{n}{fig}\PY{p}{,} \PY{n}{ax} \PY{o}{=} \PY{n}{plt}\PY{o}{.}\PY{n}{subplots}\PY{p}{(}\PY{l+m+mi}{1}\PY{p}{,} \PY{l+m+mi}{1}\PY{p}{,} \PY{n}{figsize}\PY{o}{=}\PY{p}{(}\PY{l+m+mi}{12}\PY{p}{,} \PY{l+m+mi}{6}\PY{p}{)}\PY{p}{)}
\PY{n}{fmt} \PY{o}{=} \PY{n}{ticker}\PY{o}{.}\PY{n}{FuncFormatter}\PY{p}{(}\PY{k}{lambda} \PY{n}{x}\PY{p}{,} \PY{n}{pos}\PY{p}{:} \PY{l+s+sa}{f}\PY{l+s+s1}{\PYZsq{}}\PY{l+s+si}{\PYZob{}}\PY{n}{x}\PY{o}{/}\PY{l+m+mi}{1000}\PY{l+s+si}{:}\PY{l+s+s1}{.0f}\PY{l+s+si}{\PYZcb{}}\PY{l+s+s1}{ tys.}\PY{l+s+s1}{\PYZsq{}}\PY{p}{)}
\PY{n}{sns}\PY{o}{.}\PY{n}{histplot}\PY{p}{(}\PY{n}{df}\PY{p}{[}\PY{l+s+s1}{\PYZsq{}}\PY{l+s+s1}{price}\PY{l+s+s1}{\PYZsq{}}\PY{p}{]}\PY{p}{,} \PY{n}{bins}\PY{o}{=}\PY{l+m+mi}{50}\PY{p}{,} \PY{n}{kde}\PY{o}{=}\PY{k+kc}{True}\PY{p}{,} \PY{n}{ax}\PY{o}{=}\PY{n}{ax}\PY{p}{,} \PY{n}{color}\PY{o}{=}\PY{l+s+s1}{\PYZsq{}}\PY{l+s+s1}{skyblue}\PY{l+s+s1}{\PYZsq{}}\PY{p}{)}
\PY{n}{ax}\PY{o}{.}\PY{n}{set\PYZus{}title}\PY{p}{(}\PY{l+s+s1}{\PYZsq{}}\PY{l+s+s1}{Histogram cen mieszkań po usunięciu wartości odstających – osobno dla każdego miasta}\PY{l+s+s1}{\PYZsq{}}\PY{p}{)}
\PY{n}{ax}\PY{o}{.}\PY{n}{set\PYZus{}xlabel}\PY{p}{(}\PY{l+s+s1}{\PYZsq{}}\PY{l+s+s1}{Cena (PLN)}\PY{l+s+s1}{\PYZsq{}}\PY{p}{)}
\PY{n}{ax}\PY{o}{.}\PY{n}{set\PYZus{}ylabel}\PY{p}{(}\PY{l+s+s1}{\PYZsq{}}\PY{l+s+s1}{Liczba mieszkań}\PY{l+s+s1}{\PYZsq{}}\PY{p}{)}
\PY{n}{ax}\PY{o}{.}\PY{n}{xaxis}\PY{o}{.}\PY{n}{set\PYZus{}major\PYZus{}formatter}\PY{p}{(}\PY{n}{fmt}\PY{p}{)}
\PY{n}{ax}\PY{o}{.}\PY{n}{set\PYZus{}xlim}\PY{p}{(}\PY{n}{left}\PY{o}{=}\PY{l+m+mi}{0}\PY{p}{)}
\PY{n}{ax}\PY{o}{.}\PY{n}{grid}\PY{p}{(}\PY{k+kc}{True}\PY{p}{,} \PY{n}{linestyle}\PY{o}{=}\PY{l+s+s1}{\PYZsq{}}\PY{l+s+s1}{\PYZhy{}\PYZhy{}}\PY{l+s+s1}{\PYZsq{}}\PY{p}{,} \PY{n}{alpha}\PY{o}{=}\PY{l+m+mf}{0.5}\PY{p}{)}
\PY{n}{plt}\PY{o}{.}\PY{n}{tight\PYZus{}layout}\PY{p}{(}\PY{p}{)}
\PY{n}{plt}\PY{o}{.}\PY{n}{show}\PY{p}{(}\PY{p}{)}

\PY{c+c1}{\PYZsh{} Boxplot po oczyszczeniu}
\PY{n}{fig}\PY{p}{,} \PY{n}{ax} \PY{o}{=} \PY{n}{plt}\PY{o}{.}\PY{n}{subplots}\PY{p}{(}\PY{l+m+mi}{1}\PY{p}{,} \PY{l+m+mi}{1}\PY{p}{,} \PY{n}{figsize}\PY{o}{=}\PY{p}{(}\PY{l+m+mi}{12}\PY{p}{,} \PY{l+m+mi}{6}\PY{p}{)}\PY{p}{)}
\PY{n}{sns}\PY{o}{.}\PY{n}{boxplot}\PY{p}{(}\PY{n}{x}\PY{o}{=}\PY{n}{df}\PY{p}{[}\PY{l+s+s1}{\PYZsq{}}\PY{l+s+s1}{price}\PY{l+s+s1}{\PYZsq{}}\PY{p}{]}\PY{p}{,} \PY{n}{ax}\PY{o}{=}\PY{n}{ax}\PY{p}{,} \PY{n}{color}\PY{o}{=}\PY{l+s+s1}{\PYZsq{}}\PY{l+s+s1}{skyblue}\PY{l+s+s1}{\PYZsq{}}\PY{p}{)}
\PY{n}{ax}\PY{o}{.}\PY{n}{set\PYZus{}title}\PY{p}{(}\PY{l+s+s1}{\PYZsq{}}\PY{l+s+s1}{Rozkład cen mieszkań po usunięciu wartości odstających}\PY{l+s+s1}{\PYZsq{}}\PY{p}{)}
\PY{n}{ax}\PY{o}{.}\PY{n}{set\PYZus{}xlabel}\PY{p}{(}\PY{l+s+s1}{\PYZsq{}}\PY{l+s+s1}{Cena (PLN)}\PY{l+s+s1}{\PYZsq{}}\PY{p}{)}
\PY{n}{ax}\PY{o}{.}\PY{n}{xaxis}\PY{o}{.}\PY{n}{set\PYZus{}major\PYZus{}formatter}\PY{p}{(}\PY{n}{fmt}\PY{p}{)}
\PY{n}{plt}\PY{o}{.}\PY{n}{tight\PYZus{}layout}\PY{p}{(}\PY{p}{)}
\PY{n}{plt}\PY{o}{.}\PY{n}{show}\PY{p}{(}\PY{p}{)}
\end{Verbatim}
\end{tcolorbox}

    \begin{center}
    \adjustimage{max size={0.9\linewidth}{0.9\paperheight}}{Analiza_mieszkan_Final_files/Analiza_mieszkan_Final_14_0.png}
    \end{center}
    { \hspace*{\fill} \\}
    
    \begin{center}
    \adjustimage{max size={0.9\linewidth}{0.9\paperheight}}{Analiza_mieszkan_Final_files/Analiza_mieszkan_Final_14_1.png}
    \end{center}
    { \hspace*{\fill} \\}
    
    Ze względu na duże różnice cen mieszkań pomiędzy miastami (np. Warszawa
vs.~Rzeszów), zdecydowaliśmy się na bardziej precyzyjne podejście do
filtrowania wartości odstających -- zrobiliśmy to \textbf{osobno dla
każdego miasta}.

Dla każdej lokalizacji odcięliśmy 1\% najtańszych oraz 5\% najdroższych
mieszkań (czyli zastosowaliśmy 1. i 95. percentyl lokalnie). Taki zabieg
pozwala zachować reprezentatywność danych, a jednocześnie usuwa
ekstremalne przypadki, które mogłyby zaburzyć analizę.

Dzięki temu:

\begin{itemize}
\tightlist
\item
  Unikamy błędu polegającego na wspólnym usunięciu danych według jednego
  progu cenowego.
\item
  Zachowujemy strukturę rynków lokalnych.
\item
  Histogram rozkładu cen staje się bardziej symetryczny.
\end{itemize}

\emph{Uwaga:} W wyniku tej operacji liczba obserwacji zmniejszyła się z
\textasciitilde92\,967 do \textasciitilde83\,000. Pozostałe dane dobrze
reprezentują lokalne rynki mieszkań -- bez skrajnych przypadków ani
luksusowych nieruchomości.

Po oczyszczeniu zbioru danych zaobserwowano zmianę rozkładu cen --
zredukowano liczbę ekstremalnych wartości, dzięki czemu analiza staje
się bardziej reprezentatywna, a histogram staje się wyraźnie mniej
skośny, choć wciąż lekko prawostronny. Niestety, ale wykluczenie części
obserwacji może prowadzić do utraty niektórych informacji, np. o
nieruchomościach luksusowych. My jednak skupiamy się na segmencie
mieszkań przeznaczonych dla zwykłego ``Kowalskiego''.

    \subsection{WIZUALIZACJE}\label{wizualizacje}

    \begin{tcolorbox}[breakable, size=fbox, boxrule=1pt, pad at break*=1mm,colback=cellbackground, colframe=cellborder]
\prompt{In}{incolor}{35}{\boxspacing}
\begin{Verbatim}[commandchars=\\\{\}]
\PY{c+c1}{\PYZsh{} Wykres 1: Rozkład cen mieszkań w miastach}
\PY{n}{city\PYZus{}order} \PY{o}{=} \PY{p}{(}
    \PY{n}{df}\PY{o}{.}\PY{n}{groupby}\PY{p}{(}\PY{l+s+s1}{\PYZsq{}}\PY{l+s+s1}{city}\PY{l+s+s1}{\PYZsq{}}\PY{p}{)}\PY{p}{[}\PY{l+s+s1}{\PYZsq{}}\PY{l+s+s1}{price}\PY{l+s+s1}{\PYZsq{}}\PY{p}{]}
      \PY{o}{.}\PY{n}{median}\PY{p}{(}\PY{p}{)}
      \PY{o}{.}\PY{n}{sort\PYZus{}values}\PY{p}{(}\PY{n}{ascending}\PY{o}{=}\PY{k+kc}{False}\PY{p}{)}
      \PY{o}{.}\PY{n}{index}
\PY{p}{)}

\PY{n}{fig}\PY{p}{,} \PY{n}{ax} \PY{o}{=} \PY{n}{plt}\PY{o}{.}\PY{n}{subplots}\PY{p}{(}\PY{n}{figsize}\PY{o}{=}\PY{p}{(}\PY{l+m+mi}{16}\PY{p}{,} \PY{l+m+mi}{8}\PY{p}{)}\PY{p}{)}

\PY{n}{sns}\PY{o}{.}\PY{n}{boxplot}\PY{p}{(}
    \PY{n}{x}\PY{o}{=}\PY{l+s+s1}{\PYZsq{}}\PY{l+s+s1}{city}\PY{l+s+s1}{\PYZsq{}}\PY{p}{,}
    \PY{n}{y}\PY{o}{=}\PY{l+s+s1}{\PYZsq{}}\PY{l+s+s1}{price}\PY{l+s+s1}{\PYZsq{}}\PY{p}{,}
    \PY{n}{data}\PY{o}{=}\PY{n}{df}\PY{p}{,}
    \PY{n}{order}\PY{o}{=}\PY{n}{city\PYZus{}order}\PY{p}{,}
    \PY{n}{palette}\PY{o}{=}\PY{l+s+s1}{\PYZsq{}}\PY{l+s+s1}{Set3}\PY{l+s+s1}{\PYZsq{}}\PY{p}{,}
    \PY{n}{flierprops}\PY{o}{=}\PY{n+nb}{dict}\PY{p}{(}\PY{n}{marker}\PY{o}{=}\PY{l+s+s1}{\PYZsq{}}\PY{l+s+s1}{o}\PY{l+s+s1}{\PYZsq{}}\PY{p}{,} \PY{n}{markersize}\PY{o}{=}\PY{l+m+mi}{3}\PY{p}{,} \PY{n}{markerfacecolor}\PY{o}{=}\PY{l+s+s1}{\PYZsq{}}\PY{l+s+s1}{grey}\PY{l+s+s1}{\PYZsq{}}\PY{p}{,} \PY{n}{alpha}\PY{o}{=}\PY{l+m+mf}{0.4}\PY{p}{)}
\PY{p}{)}

\PY{n}{ax}\PY{o}{.}\PY{n}{set\PYZus{}title}\PY{p}{(}\PY{l+s+s1}{\PYZsq{}}\PY{l+s+s1}{Rozkład cen mieszkań w największych miastach Polski}\PY{l+s+s1}{\PYZsq{}}\PY{p}{,} \PY{n}{fontsize}\PY{o}{=}\PY{l+m+mi}{16}\PY{p}{)}
\PY{n}{ax}\PY{o}{.}\PY{n}{set\PYZus{}xlabel}\PY{p}{(}\PY{l+s+s1}{\PYZsq{}}\PY{l+s+s1}{Miasto}\PY{l+s+s1}{\PYZsq{}}\PY{p}{)}
\PY{n}{ax}\PY{o}{.}\PY{n}{set\PYZus{}ylabel}\PY{p}{(}\PY{l+s+s1}{\PYZsq{}}\PY{l+s+s1}{Cena (PLN)}\PY{l+s+s1}{\PYZsq{}}\PY{p}{)}
\PY{n}{ax}\PY{o}{.}\PY{n}{xaxis}\PY{o}{.}\PY{n}{set\PYZus{}tick\PYZus{}params}\PY{p}{(}\PY{n}{rotation}\PY{o}{=}\PY{l+m+mi}{30}\PY{p}{)}
\PY{n}{ax}\PY{o}{.}\PY{n}{yaxis}\PY{o}{.}\PY{n}{set\PYZus{}major\PYZus{}formatter}\PY{p}{(}\PY{n}{ticker}\PY{o}{.}\PY{n}{FuncFormatter}\PY{p}{(}\PY{k}{lambda} \PY{n}{x}\PY{p}{,} \PY{n}{pos}\PY{p}{:} \PY{l+s+sa}{f}\PY{l+s+s1}{\PYZsq{}}\PY{l+s+si}{\PYZob{}}\PY{n}{x}\PY{o}{/}\PY{l+m+mi}{1000}\PY{l+s+si}{:}\PY{l+s+s1}{.0f}\PY{l+s+si}{\PYZcb{}}\PY{l+s+s1}{ tys.}\PY{l+s+s1}{\PYZsq{}}\PY{p}{)}\PY{p}{)}
\PY{n}{ax}\PY{o}{.}\PY{n}{grid}\PY{p}{(}\PY{k+kc}{True}\PY{p}{,} \PY{n}{axis}\PY{o}{=}\PY{l+s+s1}{\PYZsq{}}\PY{l+s+s1}{y}\PY{l+s+s1}{\PYZsq{}}\PY{p}{,} \PY{n}{linestyle}\PY{o}{=}\PY{l+s+s1}{\PYZsq{}}\PY{l+s+s1}{\PYZhy{}\PYZhy{}}\PY{l+s+s1}{\PYZsq{}}\PY{p}{,} \PY{n}{alpha}\PY{o}{=}\PY{l+m+mf}{0.5}\PY{p}{)}

\PY{n}{plt}\PY{o}{.}\PY{n}{tight\PYZus{}layout}\PY{p}{(}\PY{p}{)}
\PY{n}{plt}\PY{o}{.}\PY{n}{show}\PY{p}{(}\PY{p}{)}
\end{Verbatim}
\end{tcolorbox}

    \begin{Verbatim}[commandchars=\\\{\}]
C:\textbackslash{}Users\textbackslash{}olale\textbackslash{}AppData\textbackslash{}Local\textbackslash{}Temp\textbackslash{}ipykernel\_12048\textbackslash{}2364514622.py:11:
FutureWarning:

Passing `palette` without assigning `hue` is deprecated and will be removed in
v0.14.0. Assign the `x` variable to `hue` and set `legend=False` for the same
effect.

  sns.boxplot(
    \end{Verbatim}

    \begin{center}
    \adjustimage{max size={0.9\linewidth}{0.9\paperheight}}{Analiza_mieszkan_Final_files/Analiza_mieszkan_Final_17_1.png}
    \end{center}
    { \hspace*{\fill} \\}
    
    \paragraph{Rozkład cen mieszkań w największych miastach
Polski}\label{rozkux142ad-cen-mieszkaux144-w-najwiux119kszych-miastach-polski}

Boxplot jasno pokazuje duże zróżnicowanie mediany cen między miastami:
Warszawa zdecydowanie prowadzi, Rzeszów, Lublin i Szczecin plasują się w
środku stawki, natomiast najniższą medianę cen możemy dostrzec w
Radomimu i Częstochowie. Rozstępy między kwartylami są szersze w
metropoliach, co świadczy o bardziej zróżnicowanym segmencie rynkowym.
Warto zauważyć, że nawet po odcięciu 5\,\% skrajnych cen, w Warszawie,
Krakowie, Gdańsku i Gdyni widoczne są pojedyncze obserwacje powyżej
1,2\,mln PLN, co odpowiada segmentowi premium.

    \begin{tcolorbox}[breakable, size=fbox, boxrule=1pt, pad at break*=1mm,colback=cellbackground, colframe=cellborder]
\prompt{In}{incolor}{36}{\boxspacing}
\begin{Verbatim}[commandchars=\\\{\}]
\PY{c+c1}{\PYZsh{} Wykres 2: Zależność ceny od metrażu}
\PY{c+c1}{\PYZsh{} Sample do wykresu rozrzutu, cały zbiór do linii regresji i korelacji}
\PY{n}{scatter\PYZus{}sample} \PY{o}{=} \PY{n}{df}\PY{o}{.}\PY{n}{sample}\PY{p}{(}\PY{l+m+mi}{6000}\PY{p}{,} \PY{n}{random\PYZus{}state}\PY{o}{=}\PY{l+m+mi}{42}\PY{p}{)}

\PY{c+c1}{\PYZsh{} Współczynnik korelacji (Pearson)}
\PY{n}{corr} \PY{o}{=} \PY{n}{df}\PY{p}{[}\PY{l+s+s1}{\PYZsq{}}\PY{l+s+s1}{squareMeters}\PY{l+s+s1}{\PYZsq{}}\PY{p}{]}\PY{o}{.}\PY{n}{corr}\PY{p}{(}\PY{n}{df}\PY{p}{[}\PY{l+s+s1}{\PYZsq{}}\PY{l+s+s1}{price}\PY{l+s+s1}{\PYZsq{}}\PY{p}{]}\PY{p}{)}
\PY{n+nb}{print}\PY{p}{(}\PY{l+s+sa}{f}\PY{l+s+s2}{\PYZdq{}}\PY{l+s+s2}{Pearson r = }\PY{l+s+si}{\PYZob{}}\PY{n}{corr}\PY{l+s+si}{:}\PY{l+s+s2}{.3f}\PY{l+s+si}{\PYZcb{}}\PY{l+s+s2}{\PYZdq{}}\PY{p}{)}

\PY{n}{fig}\PY{p}{,} \PY{n}{ax} \PY{o}{=} \PY{n}{plt}\PY{o}{.}\PY{n}{subplots}\PY{p}{(}\PY{n}{figsize}\PY{o}{=}\PY{p}{(}\PY{l+m+mi}{13}\PY{p}{,} \PY{l+m+mi}{8}\PY{p}{)}\PY{p}{)}

\PY{n}{sns}\PY{o}{.}\PY{n}{scatterplot}\PY{p}{(}
    \PY{n}{x}\PY{o}{=}\PY{l+s+s1}{\PYZsq{}}\PY{l+s+s1}{squareMeters}\PY{l+s+s1}{\PYZsq{}}\PY{p}{,} \PY{n}{y}\PY{o}{=}\PY{l+s+s1}{\PYZsq{}}\PY{l+s+s1}{price}\PY{l+s+s1}{\PYZsq{}}\PY{p}{,}
    \PY{n}{data}\PY{o}{=}\PY{n}{scatter\PYZus{}sample}\PY{p}{,}
    \PY{n}{alpha}\PY{o}{=}\PY{l+m+mf}{0.35}\PY{p}{,} \PY{n}{color}\PY{o}{=}\PY{l+s+s1}{\PYZsq{}}\PY{l+s+s1}{royalblue}\PY{l+s+s1}{\PYZsq{}}\PY{p}{,} \PY{n}{ax}\PY{o}{=}\PY{n}{ax}
\PY{p}{)}

\PY{c+c1}{\PYZsh{} Linia trendu (regresja liniowa)}
\PY{n}{sns}\PY{o}{.}\PY{n}{regplot}\PY{p}{(}
    \PY{n}{x}\PY{o}{=}\PY{l+s+s1}{\PYZsq{}}\PY{l+s+s1}{squareMeters}\PY{l+s+s1}{\PYZsq{}}\PY{p}{,} \PY{n}{y}\PY{o}{=}\PY{l+s+s1}{\PYZsq{}}\PY{l+s+s1}{price}\PY{l+s+s1}{\PYZsq{}}\PY{p}{,}
    \PY{n}{data}\PY{o}{=}\PY{n}{df}\PY{p}{,}
    \PY{n}{scatter}\PY{o}{=}\PY{k+kc}{False}\PY{p}{,} \PY{n}{ax}\PY{o}{=}\PY{n}{ax}\PY{p}{,}
    \PY{n}{color}\PY{o}{=}\PY{l+s+s1}{\PYZsq{}}\PY{l+s+s1}{darkorange}\PY{l+s+s1}{\PYZsq{}}\PY{p}{,} \PY{n}{line\PYZus{}kws}\PY{o}{=}\PY{p}{\PYZob{}}\PY{l+s+s1}{\PYZsq{}}\PY{l+s+s1}{linestyle}\PY{l+s+s1}{\PYZsq{}}\PY{p}{:} \PY{l+s+s1}{\PYZsq{}}\PY{l+s+s1}{\PYZhy{}\PYZhy{}}\PY{l+s+s1}{\PYZsq{}}\PY{p}{\PYZcb{}}
\PY{p}{)}

\PY{n}{ax}\PY{o}{.}\PY{n}{set\PYZus{}title}\PY{p}{(}\PY{l+s+s1}{\PYZsq{}}\PY{l+s+s1}{Zależność ceny mieszkania od metrażu}\PY{l+s+s1}{\PYZsq{}}\PY{p}{,} \PY{n}{fontsize}\PY{o}{=}\PY{l+m+mi}{16}\PY{p}{)}
\PY{n}{ax}\PY{o}{.}\PY{n}{set\PYZus{}xlabel}\PY{p}{(}\PY{l+s+s1}{\PYZsq{}}\PY{l+s+s1}{Metraż (m²)}\PY{l+s+s1}{\PYZsq{}}\PY{p}{)}
\PY{n}{ax}\PY{o}{.}\PY{n}{set\PYZus{}ylabel}\PY{p}{(}\PY{l+s+s1}{\PYZsq{}}\PY{l+s+s1}{Cena (PLN)}\PY{l+s+s1}{\PYZsq{}}\PY{p}{)}
\PY{n}{ax}\PY{o}{.}\PY{n}{yaxis}\PY{o}{.}\PY{n}{set\PYZus{}major\PYZus{}formatter}\PY{p}{(}\PY{n}{ticker}\PY{o}{.}\PY{n}{FuncFormatter}\PY{p}{(}\PY{k}{lambda} \PY{n}{x}\PY{p}{,} \PY{n}{pos}\PY{p}{:} \PY{l+s+sa}{f}\PY{l+s+s1}{\PYZsq{}}\PY{l+s+si}{\PYZob{}}\PY{n}{x}\PY{o}{/}\PY{l+m+mi}{1000}\PY{l+s+si}{:}\PY{l+s+s1}{.0f}\PY{l+s+si}{\PYZcb{}}\PY{l+s+s1}{ tys.}\PY{l+s+s1}{\PYZsq{}}\PY{p}{)}\PY{p}{)}
\PY{n}{ax}\PY{o}{.}\PY{n}{grid}\PY{p}{(}\PY{k+kc}{True}\PY{p}{,} \PY{n}{linestyle}\PY{o}{=}\PY{l+s+s1}{\PYZsq{}}\PY{l+s+s1}{\PYZhy{}\PYZhy{}}\PY{l+s+s1}{\PYZsq{}}\PY{p}{,} \PY{n}{alpha}\PY{o}{=}\PY{l+m+mf}{0.4}\PY{p}{)}

\PY{c+c1}{\PYZsh{} Ewentualnie skala log10 na osi Y}
\PY{c+c1}{\PYZsh{} ax.set\PYZus{}yscale(\PYZsq{}log\PYZsq{})}

\PY{c+c1}{\PYZsh{} Podpis korelacji w prawym górnym rogu}
\PY{n}{ax}\PY{o}{.}\PY{n}{text}\PY{p}{(}
    \PY{l+m+mf}{0.97}\PY{p}{,} \PY{l+m+mf}{0.95}\PY{p}{,}
    \PY{l+s+sa}{f}\PY{l+s+s1}{\PYZsq{}}\PY{l+s+s1}{Pearson r = }\PY{l+s+si}{\PYZob{}}\PY{n}{corr}\PY{l+s+si}{:}\PY{l+s+s1}{.2f}\PY{l+s+si}{\PYZcb{}}\PY{l+s+s1}{\PYZsq{}}\PY{p}{,}
    \PY{n}{transform}\PY{o}{=}\PY{n}{ax}\PY{o}{.}\PY{n}{transAxes}\PY{p}{,}
    \PY{n}{ha}\PY{o}{=}\PY{l+s+s1}{\PYZsq{}}\PY{l+s+s1}{right}\PY{l+s+s1}{\PYZsq{}}\PY{p}{,} \PY{n}{va}\PY{o}{=}\PY{l+s+s1}{\PYZsq{}}\PY{l+s+s1}{top}\PY{l+s+s1}{\PYZsq{}}\PY{p}{,}
    \PY{n}{fontsize}\PY{o}{=}\PY{l+m+mi}{11}\PY{p}{,} \PY{n}{bbox}\PY{o}{=}\PY{n+nb}{dict}\PY{p}{(}\PY{n}{boxstyle}\PY{o}{=}\PY{l+s+s1}{\PYZsq{}}\PY{l+s+s1}{round,pad=0.3}\PY{l+s+s1}{\PYZsq{}}\PY{p}{,} \PY{n}{fc}\PY{o}{=}\PY{l+s+s1}{\PYZsq{}}\PY{l+s+s1}{white}\PY{l+s+s1}{\PYZsq{}}\PY{p}{,} \PY{n}{ec}\PY{o}{=}\PY{l+s+s1}{\PYZsq{}}\PY{l+s+s1}{gray}\PY{l+s+s1}{\PYZsq{}}\PY{p}{,} \PY{n}{alpha}\PY{o}{=}\PY{l+m+mf}{0.7}\PY{p}{)}
\PY{p}{)}

\PY{n}{plt}\PY{o}{.}\PY{n}{tight\PYZus{}layout}\PY{p}{(}\PY{p}{)}
\PY{n}{plt}\PY{o}{.}\PY{n}{show}\PY{p}{(}\PY{p}{)}
\end{Verbatim}
\end{tcolorbox}

    \begin{Verbatim}[commandchars=\\\{\}]
Pearson r = 0.563
    \end{Verbatim}

    \begin{center}
    \adjustimage{max size={0.9\linewidth}{0.9\paperheight}}{Analiza_mieszkan_Final_files/Analiza_mieszkan_Final_19_1.png}
    \end{center}
    { \hspace*{\fill} \\}
    
    \paragraph{Zależność ceny od
metrażu}\label{zaleux17cnoux15bux107-ceny-od-metraux17cu}

Wykres punktowy (na losowej próbce 6\,000 mieszkań) pokazuje wyraźną
dodatnią korelację pomiędzy metrażem a ceną. Współczynnik
\textbf{Pearsona r ≈\,0,5} wskazuje na silną zależność liniową --
szczególnie przy większych metrażach widać rosnące zróżnicowanie cen.

\textbf{Warto odnotować:} Mieszkania ≈\,40--60\,m² tworzą najbardziej
gęsty „rdzeń'' rynku, a powyżej 80\,m² rozrzut cen znacząco rośnie, co
może sugerować różne standardy wykończenia lub lokalizacje premium.

Linia trendu (pomarańczowa, regresja liniowa) dobrze oddaje medianę
zależności, lecz nie oddaje rosnącej wariancji dla dużych metraży.

    \begin{tcolorbox}[breakable, size=fbox, boxrule=1pt, pad at break*=1mm,colback=cellbackground, colframe=cellborder]
\prompt{In}{incolor}{37}{\boxspacing}
\begin{Verbatim}[commandchars=\\\{\}]
\PY{c+c1}{\PYZsh{} Wykres 3: Wpływ liczby pokoi na cenę}
\PY{n}{df}\PY{p}{[}\PY{l+s+s1}{\PYZsq{}}\PY{l+s+s1}{rooms}\PY{l+s+s1}{\PYZsq{}}\PY{p}{]} \PY{o}{=} \PY{n}{df}\PY{p}{[}\PY{l+s+s1}{\PYZsq{}}\PY{l+s+s1}{rooms}\PY{l+s+s1}{\PYZsq{}}\PY{p}{]}\PY{o}{.}\PY{n}{astype}\PY{p}{(}\PY{n+nb}{int}\PY{p}{)}

\PY{c+c1}{\PYZsh{} Posortowana lista liczby pokoi (1..6)}
\PY{n}{room\PYZus{}order} \PY{o}{=} \PY{n+nb}{sorted}\PY{p}{(}\PY{n}{df}\PY{p}{[}\PY{l+s+s1}{\PYZsq{}}\PY{l+s+s1}{rooms}\PY{l+s+s1}{\PYZsq{}}\PY{p}{]}\PY{o}{.}\PY{n}{unique}\PY{p}{(}\PY{p}{)}\PY{p}{)}

\PY{n}{fig}\PY{p}{,} \PY{n}{ax} \PY{o}{=} \PY{n}{plt}\PY{o}{.}\PY{n}{subplots}\PY{p}{(}\PY{n}{figsize}\PY{o}{=}\PY{p}{(}\PY{l+m+mi}{12}\PY{p}{,} \PY{l+m+mi}{8}\PY{p}{)}\PY{p}{)}

\PY{n}{sns}\PY{o}{.}\PY{n}{boxplot}\PY{p}{(}
    \PY{n}{x}\PY{o}{=}\PY{l+s+s1}{\PYZsq{}}\PY{l+s+s1}{rooms}\PY{l+s+s1}{\PYZsq{}}\PY{p}{,}
    \PY{n}{y}\PY{o}{=}\PY{l+s+s1}{\PYZsq{}}\PY{l+s+s1}{price}\PY{l+s+s1}{\PYZsq{}}\PY{p}{,}
    \PY{n}{data}\PY{o}{=}\PY{n}{df}\PY{p}{,}
    \PY{n}{order}\PY{o}{=}\PY{n}{room\PYZus{}order}\PY{p}{,}          
    \PY{n}{palette}\PY{o}{=}\PY{l+s+s1}{\PYZsq{}}\PY{l+s+s1}{viridis}\PY{l+s+s1}{\PYZsq{}}\PY{p}{,}           
    \PY{n}{ax}\PY{o}{=}\PY{n}{ax}
\PY{p}{)}

\PY{c+c1}{\PYZsh{} Tytuły i opisy osi}
\PY{n}{ax}\PY{o}{.}\PY{n}{set\PYZus{}title}\PY{p}{(}\PY{l+s+s1}{\PYZsq{}}\PY{l+s+s1}{Wpływ liczby pokoi na cenę mieszkania}\PY{l+s+s1}{\PYZsq{}}\PY{p}{,} \PY{n}{fontsize}\PY{o}{=}\PY{l+m+mi}{16}\PY{p}{)}
\PY{n}{ax}\PY{o}{.}\PY{n}{set\PYZus{}xlabel}\PY{p}{(}\PY{l+s+s1}{\PYZsq{}}\PY{l+s+s1}{Liczba pokoi}\PY{l+s+s1}{\PYZsq{}}\PY{p}{)}
\PY{n}{ax}\PY{o}{.}\PY{n}{set\PYZus{}ylabel}\PY{p}{(}\PY{l+s+s1}{\PYZsq{}}\PY{l+s+s1}{Cena (PLN)}\PY{l+s+s1}{\PYZsq{}}\PY{p}{)}
\PY{n}{ax}\PY{o}{.}\PY{n}{yaxis}\PY{o}{.}\PY{n}{set\PYZus{}major\PYZus{}formatter}\PY{p}{(}\PY{n}{ticker}\PY{o}{.}\PY{n}{FuncFormatter}\PY{p}{(}\PY{k}{lambda} \PY{n}{x}\PY{p}{,} \PY{n}{pos}\PY{p}{:} \PY{l+s+sa}{f}\PY{l+s+s1}{\PYZsq{}}\PY{l+s+si}{\PYZob{}}\PY{n}{x}\PY{o}{/}\PY{l+m+mi}{1000}\PY{l+s+si}{:}\PY{l+s+s1}{.0f}\PY{l+s+si}{\PYZcb{}}\PY{l+s+s1}{ tys.}\PY{l+s+s1}{\PYZsq{}}\PY{p}{)}\PY{p}{)}
\PY{n}{ax}\PY{o}{.}\PY{n}{grid}\PY{p}{(}\PY{k+kc}{True}\PY{p}{,} \PY{n}{axis}\PY{o}{=}\PY{l+s+s1}{\PYZsq{}}\PY{l+s+s1}{y}\PY{l+s+s1}{\PYZsq{}}\PY{p}{,} \PY{n}{linestyle}\PY{o}{=}\PY{l+s+s1}{\PYZsq{}}\PY{l+s+s1}{\PYZhy{}\PYZhy{}}\PY{l+s+s1}{\PYZsq{}}\PY{p}{,} \PY{n}{alpha}\PY{o}{=}\PY{l+m+mf}{0.4}\PY{p}{)}

\PY{c+c1}{\PYZsh{} Dodanie etykiet mediany nad każdym boxem}
\PY{k}{for} \PY{n}{i}\PY{p}{,} \PY{n}{rooms} \PY{o+ow}{in} \PY{n+nb}{enumerate}\PY{p}{(}\PY{n}{room\PYZus{}order}\PY{p}{)}\PY{p}{:}
    \PY{n}{median\PYZus{}price} \PY{o}{=} \PY{n}{df}\PY{p}{[}\PY{n}{df}\PY{p}{[}\PY{l+s+s1}{\PYZsq{}}\PY{l+s+s1}{rooms}\PY{l+s+s1}{\PYZsq{}}\PY{p}{]} \PY{o}{==} \PY{n}{rooms}\PY{p}{]}\PY{p}{[}\PY{l+s+s1}{\PYZsq{}}\PY{l+s+s1}{price}\PY{l+s+s1}{\PYZsq{}}\PY{p}{]}\PY{o}{.}\PY{n}{median}\PY{p}{(}\PY{p}{)}
    \PY{n}{ax}\PY{o}{.}\PY{n}{annotate}\PY{p}{(}\PY{l+s+sa}{f}\PY{l+s+s1}{\PYZsq{}}\PY{l+s+si}{\PYZob{}}\PY{n}{median\PYZus{}price}\PY{o}{/}\PY{l+m+mi}{1000}\PY{l+s+si}{:}\PY{l+s+s1}{.0f}\PY{l+s+si}{\PYZcb{}}\PY{l+s+s1}{k}\PY{l+s+s1}{\PYZsq{}}\PY{p}{,} \PY{n}{xy}\PY{o}{=}\PY{p}{(}\PY{n}{i}\PY{p}{,} \PY{n}{median\PYZus{}price}\PY{p}{)}\PY{p}{,}
                \PY{n}{xytext}\PY{o}{=}\PY{p}{(}\PY{l+m+mi}{0}\PY{p}{,} \PY{l+m+mi}{8}\PY{p}{)}\PY{p}{,} \PY{n}{textcoords}\PY{o}{=}\PY{l+s+s1}{\PYZsq{}}\PY{l+s+s1}{offset points}\PY{l+s+s1}{\PYZsq{}}\PY{p}{,}
                \PY{n}{ha}\PY{o}{=}\PY{l+s+s1}{\PYZsq{}}\PY{l+s+s1}{center}\PY{l+s+s1}{\PYZsq{}}\PY{p}{,} \PY{n}{va}\PY{o}{=}\PY{l+s+s1}{\PYZsq{}}\PY{l+s+s1}{bottom}\PY{l+s+s1}{\PYZsq{}}\PY{p}{,} \PY{n}{fontsize}\PY{o}{=}\PY{l+m+mi}{10}\PY{p}{,} \PY{n}{color}\PY{o}{=}\PY{l+s+s1}{\PYZsq{}}\PY{l+s+s1}{black}\PY{l+s+s1}{\PYZsq{}}\PY{p}{)}

\PY{n}{plt}\PY{o}{.}\PY{n}{tight\PYZus{}layout}\PY{p}{(}\PY{p}{)}
\PY{n}{plt}\PY{o}{.}\PY{n}{show}\PY{p}{(}\PY{p}{)}
\end{Verbatim}
\end{tcolorbox}

    \begin{Verbatim}[commandchars=\\\{\}]
C:\textbackslash{}Users\textbackslash{}olale\textbackslash{}AppData\textbackslash{}Local\textbackslash{}Temp\textbackslash{}ipykernel\_12048\textbackslash{}15196970.py:9: FutureWarning:

Passing `palette` without assigning `hue` is deprecated and will be removed in
v0.14.0. Assign the `x` variable to `hue` and set `legend=False` for the same
effect.

  sns.boxplot(
    \end{Verbatim}

    \begin{center}
    \adjustimage{max size={0.9\linewidth}{0.9\paperheight}}{Analiza_mieszkan_Final_files/Analiza_mieszkan_Final_21_1.png}
    \end{center}
    { \hspace*{\fill} \\}
    
    \paragraph{Wpływ liczby pokoi na cenę
mieszkania}\label{wpux142yw-liczby-pokoi-na-cenux119-mieszkania}

Liczba pokoi jest jednym z kluczowych czynników wpływających na cenę
nieruchomości. Analiza boxplotów pokazuje wyraźny wzrost mediany cen
wraz ze wzrostem liczby pokoi. Wśród mieszkań 2 i 3-pokojowych
obserwujemy największą liczbę wartości odstających. Warto zauważyć, że
rozpiętość cenowa rośnie wraz z liczbą pokoi. Zaskakujące jest, że
najmniejsza różnica w medianie cem występuje między mieszkaniami
4-pokojowymi (890 tys.) a 5-pokojowymi (poniżej 990 tys.)

    \begin{tcolorbox}[breakable, size=fbox, boxrule=1pt, pad at break*=1mm,colback=cellbackground, colframe=cellborder]
\prompt{In}{incolor}{39}{\boxspacing}
\begin{Verbatim}[commandchars=\\\{\}]
\PY{c+c1}{\PYZsh{} Wykres 4: Udział mieszkań z udogodnieniami}

\PY{c+c1}{\PYZsh{} Kolumny udogodnień}
\PY{n}{amenities} \PY{o}{=} \PY{p}{[}\PY{l+s+s1}{\PYZsq{}}\PY{l+s+s1}{hasParkingSpace}\PY{l+s+s1}{\PYZsq{}}\PY{p}{,} \PY{l+s+s1}{\PYZsq{}}\PY{l+s+s1}{hasBalcony}\PY{l+s+s1}{\PYZsq{}}\PY{p}{,} \PY{l+s+s1}{\PYZsq{}}\PY{l+s+s1}{hasElevator}\PY{l+s+s1}{\PYZsq{}}\PY{p}{,}
             \PY{l+s+s1}{\PYZsq{}}\PY{l+s+s1}{hasSecurity}\PY{l+s+s1}{\PYZsq{}}\PY{p}{,} \PY{l+s+s1}{\PYZsq{}}\PY{l+s+s1}{hasStorageRoom}\PY{l+s+s1}{\PYZsq{}}\PY{p}{]}

\PY{c+c1}{\PYZsh{} Zamiana \PYZsq{}yes\PYZsq{}/\PYZsq{}no\PYZsq{} → 1/0}
\PY{n}{amenity\PYZus{}counts} \PY{o}{=} \PY{p}{(}
    \PY{n}{df}\PY{p}{[}\PY{n}{amenities}\PY{p}{]}
      \PY{o}{.}\PY{n}{replace}\PY{p}{(}\PY{p}{\PYZob{}}\PY{l+s+s1}{\PYZsq{}}\PY{l+s+s1}{yes}\PY{l+s+s1}{\PYZsq{}}\PY{p}{:} \PY{l+m+mi}{1}\PY{p}{,} \PY{l+s+s1}{\PYZsq{}}\PY{l+s+s1}{no}\PY{l+s+s1}{\PYZsq{}}\PY{p}{:} \PY{l+m+mi}{0}\PY{p}{\PYZcb{}}\PY{p}{)}
      \PY{o}{.}\PY{n}{sum}\PY{p}{(}\PY{p}{)}
      \PY{o}{.}\PY{n}{sort\PYZus{}values}\PY{p}{(}\PY{n}{ascending}\PY{o}{=}\PY{k+kc}{False}\PY{p}{)}
\PY{p}{)}

\PY{c+c1}{\PYZsh{} Wykres}
\PY{n}{fig}\PY{p}{,} \PY{n}{ax} \PY{o}{=} \PY{n}{plt}\PY{o}{.}\PY{n}{subplots}\PY{p}{(}\PY{n}{figsize}\PY{o}{=}\PY{p}{(}\PY{l+m+mi}{10}\PY{p}{,} \PY{l+m+mi}{6}\PY{p}{)}\PY{p}{)}
\PY{n}{amenity\PYZus{}counts}\PY{o}{.}\PY{n}{plot}\PY{p}{(}\PY{n}{kind}\PY{o}{=}\PY{l+s+s1}{\PYZsq{}}\PY{l+s+s1}{barh}\PY{l+s+s1}{\PYZsq{}}\PY{p}{,} \PY{n}{color}\PY{o}{=}\PY{l+s+s1}{\PYZsq{}}\PY{l+s+s1}{teal}\PY{l+s+s1}{\PYZsq{}}\PY{p}{,} \PY{n}{ax}\PY{o}{=}\PY{n}{ax}\PY{p}{)}

\PY{n}{ax}\PY{o}{.}\PY{n}{set\PYZus{}title}\PY{p}{(}\PY{l+s+s1}{\PYZsq{}}\PY{l+s+s1}{Udział mieszkań z udogodnieniami}\PY{l+s+s1}{\PYZsq{}}\PY{p}{,} \PY{n}{fontsize}\PY{o}{=}\PY{l+m+mi}{16}\PY{p}{)}
\PY{n}{ax}\PY{o}{.}\PY{n}{set\PYZus{}xlabel}\PY{p}{(}\PY{l+s+s1}{\PYZsq{}}\PY{l+s+s1}{Liczba mieszkań}\PY{l+s+s1}{\PYZsq{}}\PY{p}{)}
\PY{n}{ax}\PY{o}{.}\PY{n}{set\PYZus{}ylabel}\PY{p}{(}\PY{l+s+s1}{\PYZsq{}}\PY{l+s+s1}{Udogodnienie}\PY{l+s+s1}{\PYZsq{}}\PY{p}{)}
\PY{n}{ax}\PY{o}{.}\PY{n}{invert\PYZus{}yaxis}\PY{p}{(}\PY{p}{)}

\PY{c+c1}{\PYZsh{} Dodawanie etykiet procentowych}
\PY{n}{total\PYZus{}flats} \PY{o}{=} \PY{n+nb}{len}\PY{p}{(}\PY{n}{df}\PY{p}{)}
\PY{k}{for} \PY{n}{i}\PY{p}{,} \PY{p}{(}\PY{n}{value}\PY{p}{,} \PY{n}{name}\PY{p}{)} \PY{o+ow}{in} \PY{n+nb}{enumerate}\PY{p}{(}\PY{n+nb}{zip}\PY{p}{(}\PY{n}{amenity\PYZus{}counts}\PY{o}{.}\PY{n}{values}\PY{p}{,} \PY{n}{amenity\PYZus{}counts}\PY{o}{.}\PY{n}{index}\PY{p}{)}\PY{p}{)}\PY{p}{:}
    \PY{n}{pct} \PY{o}{=} \PY{n}{value} \PY{o}{/} \PY{n}{total\PYZus{}flats} \PY{o}{*} \PY{l+m+mi}{100}
    \PY{n}{ax}\PY{o}{.}\PY{n}{text}\PY{p}{(}\PY{n}{value} \PY{o}{+} \PY{n}{total\PYZus{}flats} \PY{o}{*} \PY{l+m+mf}{0.002}\PY{p}{,} \PY{n}{i}\PY{p}{,} \PY{l+s+sa}{f}\PY{l+s+s1}{\PYZsq{}}\PY{l+s+si}{\PYZob{}}\PY{n}{pct}\PY{l+s+si}{:}\PY{l+s+s1}{.1f}\PY{l+s+si}{\PYZcb{}}\PY{l+s+s1}{\PYZpc{}}\PY{l+s+s1}{\PYZsq{}}\PY{p}{,} \PY{n}{va}\PY{o}{=}\PY{l+s+s1}{\PYZsq{}}\PY{l+s+s1}{center}\PY{l+s+s1}{\PYZsq{}}\PY{p}{,} \PY{n}{fontsize}\PY{o}{=}\PY{l+m+mi}{10}\PY{p}{)}

\PY{n}{plt}\PY{o}{.}\PY{n}{tight\PYZus{}layout}\PY{p}{(}\PY{p}{)}
\PY{n}{plt}\PY{o}{.}\PY{n}{show}\PY{p}{(}\PY{p}{)}
\end{Verbatim}
\end{tcolorbox}

    \begin{Verbatim}[commandchars=\\\{\}]
C:\textbackslash{}Users\textbackslash{}olale\textbackslash{}AppData\textbackslash{}Local\textbackslash{}Temp\textbackslash{}ipykernel\_12048\textbackslash{}422329188.py:10:
FutureWarning: Downcasting behavior in `replace` is deprecated and will be
removed in a future version. To retain the old behavior, explicitly call
`result.infer\_objects(copy=False)`. To opt-in to the future behavior, set
`pd.set\_option('future.no\_silent\_downcasting', True)`
  .replace(\{'yes': 1, 'no': 0\})
    \end{Verbatim}

    \begin{center}
    \adjustimage{max size={0.9\linewidth}{0.9\paperheight}}{Analiza_mieszkan_Final_files/Analiza_mieszkan_Final_23_1.png}
    \end{center}
    { \hspace*{\fill} \\}
    
    \paragraph{Udział mieszkań wyposażonych w wybrane
udogodnienia}\label{udziaux142-mieszkaux144-wyposaux17conych-w-wybrane-udogodnienia}

Najczęściej spotykanym udogodnieniem jest \textbf{balkon} -- posiada go
aż \textasciitilde58\% analizowanych mieszkań. Na kolejnych miejscach
znajznajduje winda (ok.\,53\%) oraz komórka lokatorska (ok.\,45\%), co
odzwierciedla przewagę bloków i apartamentowców w większych miastach.

Z kolei \textbf{miejsce parkingowe} i \textbf{ochrona} występują
rzadziej (ok. 26\% i\,10\,\%), gdyż są charakterystyczne głównie dla
nowszych osiedli o podwyższonym standardzie.

    \begin{tcolorbox}[breakable, size=fbox, boxrule=1pt, pad at break*=1mm,colback=cellbackground, colframe=cellborder]
\prompt{In}{incolor}{40}{\boxspacing}
\begin{Verbatim}[commandchars=\\\{\}]
\PY{c+c1}{\PYZsh{} Wykres 5: Wpływ odległości od centrum na cenę za metr kwadratowy}
\PY{n}{df} \PY{o}{=} \PY{n}{df}\PY{p}{[}\PY{n}{df}\PY{p}{[}\PY{l+s+s1}{\PYZsq{}}\PY{l+s+s1}{squareMeters}\PY{l+s+s1}{\PYZsq{}}\PY{p}{]} \PY{o}{\PYZgt{}} \PY{l+m+mi}{0}\PY{p}{]}                     
\PY{n}{df}\PY{p}{[}\PY{l+s+s1}{\PYZsq{}}\PY{l+s+s1}{price\PYZus{}per\PYZus{}m2}\PY{l+s+s1}{\PYZsq{}}\PY{p}{]} \PY{o}{=} \PY{n}{df}\PY{p}{[}\PY{l+s+s1}{\PYZsq{}}\PY{l+s+s1}{price}\PY{l+s+s1}{\PYZsq{}}\PY{p}{]} \PY{o}{/} \PY{n}{df}\PY{p}{[}\PY{l+s+s1}{\PYZsq{}}\PY{l+s+s1}{squareMeters}\PY{l+s+s1}{\PYZsq{}}\PY{p}{]}

\PY{c+c1}{\PYZsh{} Scatter z kolorem wg miasta}
\PY{n}{sample} \PY{o}{=} \PY{n}{df}\PY{o}{.}\PY{n}{sample}\PY{p}{(}\PY{l+m+mi}{6000}\PY{p}{,} \PY{n}{random\PYZus{}state}\PY{o}{=}\PY{l+m+mi}{42}\PY{p}{)}

\PY{n}{fig}\PY{p}{,} \PY{n}{ax} \PY{o}{=} \PY{n}{plt}\PY{o}{.}\PY{n}{subplots}\PY{p}{(}\PY{n}{figsize}\PY{o}{=}\PY{p}{(}\PY{l+m+mi}{12}\PY{p}{,} \PY{l+m+mi}{8}\PY{p}{)}\PY{p}{)}

\PY{n}{sns}\PY{o}{.}\PY{n}{scatterplot}\PY{p}{(}
    \PY{n}{data}\PY{o}{=}\PY{n}{sample}\PY{p}{,}
    \PY{n}{x}\PY{o}{=}\PY{l+s+s1}{\PYZsq{}}\PY{l+s+s1}{centreDistance}\PY{l+s+s1}{\PYZsq{}}\PY{p}{,}
    \PY{n}{y}\PY{o}{=}\PY{l+s+s1}{\PYZsq{}}\PY{l+s+s1}{price\PYZus{}per\PYZus{}m2}\PY{l+s+s1}{\PYZsq{}}\PY{p}{,}
    \PY{n}{hue}\PY{o}{=}\PY{l+s+s1}{\PYZsq{}}\PY{l+s+s1}{city}\PY{l+s+s1}{\PYZsq{}}\PY{p}{,}
    \PY{n}{palette}\PY{o}{=}\PY{l+s+s1}{\PYZsq{}}\PY{l+s+s1}{tab10}\PY{l+s+s1}{\PYZsq{}}\PY{p}{,}
    \PY{n}{alpha}\PY{o}{=}\PY{l+m+mf}{0.3}\PY{p}{,}
    \PY{n}{ax}\PY{o}{=}\PY{n}{ax}
\PY{p}{)}

\PY{c+c1}{\PYZsh{} Linia LOWESS (lokalnie wygładzona regresja)}
\PY{n}{sns}\PY{o}{.}\PY{n}{regplot}\PY{p}{(}
    \PY{n}{data}\PY{o}{=}\PY{n}{df}\PY{p}{,} \PY{n}{x}\PY{o}{=}\PY{l+s+s1}{\PYZsq{}}\PY{l+s+s1}{centreDistance}\PY{l+s+s1}{\PYZsq{}}\PY{p}{,} \PY{n}{y}\PY{o}{=}\PY{l+s+s1}{\PYZsq{}}\PY{l+s+s1}{price\PYZus{}per\PYZus{}m2}\PY{l+s+s1}{\PYZsq{}}\PY{p}{,}
    \PY{n}{scatter}\PY{o}{=}\PY{k+kc}{False}\PY{p}{,} \PY{n}{lowess}\PY{o}{=}\PY{k+kc}{True}\PY{p}{,}
    \PY{n}{color}\PY{o}{=}\PY{l+s+s1}{\PYZsq{}}\PY{l+s+s1}{darkorange}\PY{l+s+s1}{\PYZsq{}}\PY{p}{,} \PY{n}{line\PYZus{}kws}\PY{o}{=}\PY{p}{\PYZob{}}\PY{l+s+s1}{\PYZsq{}}\PY{l+s+s1}{linestyle}\PY{l+s+s1}{\PYZsq{}}\PY{p}{:} \PY{l+s+s1}{\PYZsq{}}\PY{l+s+s1}{\PYZhy{}\PYZhy{}}\PY{l+s+s1}{\PYZsq{}}\PY{p}{\PYZcb{}}\PY{p}{,} \PY{n}{ax}\PY{o}{=}\PY{n}{ax}
\PY{p}{)}
\PY{c+c1}{\PYZsh{} Formatowanie osi}
\PY{n}{ax}\PY{o}{.}\PY{n}{set\PYZus{}title}\PY{p}{(}\PY{l+s+s1}{\PYZsq{}}\PY{l+s+s1}{Wpływ odległości od centrum na cenę za metr kwadratowy}\PY{l+s+s1}{\PYZsq{}}\PY{p}{,} \PY{n}{fontsize}\PY{o}{=}\PY{l+m+mi}{16}\PY{p}{)}
\PY{n}{ax}\PY{o}{.}\PY{n}{set\PYZus{}xlabel}\PY{p}{(}\PY{l+s+s1}{\PYZsq{}}\PY{l+s+s1}{Odległość od centrum (km)}\PY{l+s+s1}{\PYZsq{}}\PY{p}{)}
\PY{n}{ax}\PY{o}{.}\PY{n}{set\PYZus{}ylabel}\PY{p}{(}\PY{l+s+s1}{\PYZsq{}}\PY{l+s+s1}{Cena za m² (PLN)}\PY{l+s+s1}{\PYZsq{}}\PY{p}{)}
\PY{n}{ax}\PY{o}{.}\PY{n}{yaxis}\PY{o}{.}\PY{n}{set\PYZus{}major\PYZus{}formatter}\PY{p}{(}\PY{n}{ticker}\PY{o}{.}\PY{n}{FuncFormatter}\PY{p}{(}\PY{k}{lambda} \PY{n}{x}\PY{p}{,} \PY{n}{pos}\PY{p}{:} \PY{l+s+sa}{f}\PY{l+s+s1}{\PYZsq{}}\PY{l+s+si}{\PYZob{}}\PY{n}{x}\PY{o}{/}\PY{l+m+mi}{1000}\PY{l+s+si}{:}\PY{l+s+s1}{.0f}\PY{l+s+si}{\PYZcb{}}\PY{l+s+s1}{ tys.}\PY{l+s+s1}{\PYZsq{}}\PY{p}{)}\PY{p}{)}

\PY{n}{ax}\PY{o}{.}\PY{n}{grid}\PY{p}{(}\PY{k+kc}{True}\PY{p}{,} \PY{n}{linestyle}\PY{o}{=}\PY{l+s+s1}{\PYZsq{}}\PY{l+s+s1}{\PYZhy{}\PYZhy{}}\PY{l+s+s1}{\PYZsq{}}\PY{p}{,} \PY{n}{alpha}\PY{o}{=}\PY{l+m+mf}{0.4}\PY{p}{)}
\PY{n}{ax}\PY{o}{.}\PY{n}{legend}\PY{p}{(}\PY{n}{title}\PY{o}{=}\PY{l+s+s1}{\PYZsq{}}\PY{l+s+s1}{Miasto}\PY{l+s+s1}{\PYZsq{}}\PY{p}{,} \PY{n}{bbox\PYZus{}to\PYZus{}anchor}\PY{o}{=}\PY{p}{(}\PY{l+m+mf}{1.02}\PY{p}{,} \PY{l+m+mi}{1}\PY{p}{)}\PY{p}{,} \PY{n}{loc}\PY{o}{=}\PY{l+s+s1}{\PYZsq{}}\PY{l+s+s1}{upper left}\PY{l+s+s1}{\PYZsq{}}\PY{p}{)}

\PY{c+c1}{\PYZsh{} Współczynnik korelacji}
\PY{n}{pearson\PYZus{}r} \PY{o}{=} \PY{n}{df}\PY{p}{[}\PY{l+s+s1}{\PYZsq{}}\PY{l+s+s1}{centreDistance}\PY{l+s+s1}{\PYZsq{}}\PY{p}{]}\PY{o}{.}\PY{n}{corr}\PY{p}{(}\PY{n}{df}\PY{p}{[}\PY{l+s+s1}{\PYZsq{}}\PY{l+s+s1}{price\PYZus{}per\PYZus{}m2}\PY{l+s+s1}{\PYZsq{}}\PY{p}{]}\PY{p}{)}
\PY{n}{ax}\PY{o}{.}\PY{n}{text}\PY{p}{(}\PY{l+m+mf}{0.97}\PY{p}{,} \PY{l+m+mf}{0.05}\PY{p}{,} \PY{l+s+sa}{f}\PY{l+s+s1}{\PYZsq{}}\PY{l+s+s1}{Pearson r = }\PY{l+s+si}{\PYZob{}}\PY{n}{pearson\PYZus{}r}\PY{l+s+si}{:}\PY{l+s+s1}{.2f}\PY{l+s+si}{\PYZcb{}}\PY{l+s+s1}{\PYZsq{}}\PY{p}{,}
        \PY{n}{transform}\PY{o}{=}\PY{n}{ax}\PY{o}{.}\PY{n}{transAxes}\PY{p}{,} \PY{n}{ha}\PY{o}{=}\PY{l+s+s1}{\PYZsq{}}\PY{l+s+s1}{right}\PY{l+s+s1}{\PYZsq{}}\PY{p}{,} \PY{n}{va}\PY{o}{=}\PY{l+s+s1}{\PYZsq{}}\PY{l+s+s1}{bottom}\PY{l+s+s1}{\PYZsq{}}\PY{p}{,}
        \PY{n}{bbox}\PY{o}{=}\PY{n+nb}{dict}\PY{p}{(}\PY{n}{boxstyle}\PY{o}{=}\PY{l+s+s1}{\PYZsq{}}\PY{l+s+s1}{round,pad=0.3}\PY{l+s+s1}{\PYZsq{}}\PY{p}{,} \PY{n}{fc}\PY{o}{=}\PY{l+s+s1}{\PYZsq{}}\PY{l+s+s1}{white}\PY{l+s+s1}{\PYZsq{}}\PY{p}{,} \PY{n}{ec}\PY{o}{=}\PY{l+s+s1}{\PYZsq{}}\PY{l+s+s1}{gray}\PY{l+s+s1}{\PYZsq{}}\PY{p}{,} \PY{n}{alpha}\PY{o}{=}\PY{l+m+mf}{0.6}\PY{p}{)}\PY{p}{)}

\PY{n}{plt}\PY{o}{.}\PY{n}{tight\PYZus{}layout}\PY{p}{(}\PY{p}{)}
\PY{n}{plt}\PY{o}{.}\PY{n}{show}\PY{p}{(}\PY{p}{)}
\end{Verbatim}
\end{tcolorbox}

    \begin{center}
    \adjustimage{max size={0.9\linewidth}{0.9\paperheight}}{Analiza_mieszkan_Final_files/Analiza_mieszkan_Final_25_0.png}
    \end{center}
    { \hspace*{\fill} \\}
    
    \paragraph{Wpływ odległości od centrum na cenę za metr
kwadratowy}\label{wpux142yw-odlegux142oux15bci-od-centrum-na-cenux119-za-metr-kwadratowy}

Wykres rozrzutu pokazuje relację między odległością od centrum miasta a
ceną za metr kwadratowy mieszkania. Dane zostały dodatkowo pokolorowane
według miasta. Współczynnik korelacji Pearsona wynosi r = 0.07, co
oznacza bardzo słabą dodatnią korelację.

Jest to dosć zaskakujące i Wbrew intuicji, gdyż brakuje wyraźnej
zależności między odległością a ceną za m² -- możliwe, że wpływ
lokalizacji ukryty pod innymi zmiennymi.

Widoczny silny rozrzut cen na każdym poziomie odległości -- sugeruje, że
sam dystans od centrum nie jest wystarczającym predyktorem ceny.

Wykres daje ciekawy wgląd w złożoność rynku mieszkaniowego i wskazuje na
potrzebę uwzględnienia dodatkowych cech w analizie.

    \subsection{ANALIZA OPISOWA}\label{analiza-opisowa}

    \begin{tcolorbox}[breakable, size=fbox, boxrule=1pt, pad at break*=1mm,colback=cellbackground, colframe=cellborder]
\prompt{In}{incolor}{41}{\boxspacing}
\begin{Verbatim}[commandchars=\\\{\}]
\PY{c+c1}{\PYZsh{} Analiza 1: Statystyki cen za metr kwadratowy według miast}
\PY{n}{city\PYZus{}stats} \PY{o}{=} \PY{p}{(}
    \PY{n}{df}\PY{o}{.}\PY{n}{groupby}\PY{p}{(}\PY{l+s+s1}{\PYZsq{}}\PY{l+s+s1}{city}\PY{l+s+s1}{\PYZsq{}}\PY{p}{)}\PY{p}{[}\PY{l+s+s1}{\PYZsq{}}\PY{l+s+s1}{price\PYZus{}per\PYZus{}m2}\PY{l+s+s1}{\PYZsq{}}\PY{p}{]}
      \PY{o}{.}\PY{n}{agg}\PY{p}{(}\PY{n}{mean}\PY{o}{=}\PY{l+s+s1}{\PYZsq{}}\PY{l+s+s1}{mean}\PY{l+s+s1}{\PYZsq{}}\PY{p}{,} \PY{n}{median}\PY{o}{=}\PY{l+s+s1}{\PYZsq{}}\PY{l+s+s1}{median}\PY{l+s+s1}{\PYZsq{}}\PY{p}{,} \PY{n}{std}\PY{o}{=}\PY{l+s+s1}{\PYZsq{}}\PY{l+s+s1}{std}\PY{l+s+s1}{\PYZsq{}}\PY{p}{,} \PY{n}{count}\PY{o}{=}\PY{l+s+s1}{\PYZsq{}}\PY{l+s+s1}{count}\PY{l+s+s1}{\PYZsq{}}\PY{p}{)}
      \PY{o}{.}\PY{n}{sort\PYZus{}values}\PY{p}{(}\PY{n}{by}\PY{o}{=}\PY{l+s+s1}{\PYZsq{}}\PY{l+s+s1}{median}\PY{l+s+s1}{\PYZsq{}}\PY{p}{,} \PY{n}{ascending}\PY{o}{=}\PY{k+kc}{False}\PY{p}{)}
\PY{p}{)}

\PY{c+c1}{\PYZsh{} Barplot mediany cena/m2}
\PY{n}{city\PYZus{}stats\PYZus{}reset} \PY{o}{=} \PY{n}{city\PYZus{}stats}\PY{o}{.}\PY{n}{reset\PYZus{}index}\PY{p}{(}\PY{p}{)}

\PY{n}{fig}\PY{p}{,} \PY{n}{ax} \PY{o}{=} \PY{n}{plt}\PY{o}{.}\PY{n}{subplots}\PY{p}{(}\PY{n}{figsize}\PY{o}{=}\PY{p}{(}\PY{l+m+mi}{14}\PY{p}{,} \PY{l+m+mi}{8}\PY{p}{)}\PY{p}{)}
\PY{n}{sns}\PY{o}{.}\PY{n}{barplot}\PY{p}{(}
    \PY{n}{x}\PY{o}{=}\PY{l+s+s1}{\PYZsq{}}\PY{l+s+s1}{city}\PY{l+s+s1}{\PYZsq{}}\PY{p}{,} \PY{n}{y}\PY{o}{=}\PY{l+s+s1}{\PYZsq{}}\PY{l+s+s1}{median}\PY{l+s+s1}{\PYZsq{}}\PY{p}{,}
    \PY{n}{data}\PY{o}{=}\PY{n}{city\PYZus{}stats\PYZus{}reset}\PY{p}{,}
    \PY{n}{palette}\PY{o}{=}\PY{l+s+s1}{\PYZsq{}}\PY{l+s+s1}{light:\PYZsh{}5A9\PYZus{}r}\PY{l+s+s1}{\PYZsq{}}\PY{p}{,}
    \PY{n}{ax}\PY{o}{=}\PY{n}{ax}
\PY{p}{)}

\PY{n}{ax}\PY{o}{.}\PY{n}{set\PYZus{}title}\PY{p}{(}\PY{l+s+s1}{\PYZsq{}}\PY{l+s+s1}{Mediana cen za m² w największych miastach Polski}\PY{l+s+s1}{\PYZsq{}}\PY{p}{,} \PY{n}{fontsize}\PY{o}{=}\PY{l+m+mi}{16}\PY{p}{)}
\PY{n}{ax}\PY{o}{.}\PY{n}{set\PYZus{}xlabel}\PY{p}{(}\PY{l+s+s1}{\PYZsq{}}\PY{l+s+s1}{\PYZsq{}}\PY{p}{)}
\PY{n}{ax}\PY{o}{.}\PY{n}{set\PYZus{}ylabel}\PY{p}{(}\PY{l+s+s1}{\PYZsq{}}\PY{l+s+s1}{Mediana ceny za m² (PLN)}\PY{l+s+s1}{\PYZsq{}}\PY{p}{)}
\PY{n}{ax}\PY{o}{.}\PY{n}{yaxis}\PY{o}{.}\PY{n}{set\PYZus{}major\PYZus{}formatter}\PY{p}{(}\PY{n}{ticker}\PY{o}{.}\PY{n}{FuncFormatter}\PY{p}{(}\PY{k}{lambda} \PY{n}{x}\PY{p}{,} \PY{n}{pos}\PY{p}{:} \PY{l+s+sa}{f}\PY{l+s+s1}{\PYZsq{}}\PY{l+s+si}{\PYZob{}}\PY{n}{x}\PY{o}{/}\PY{l+m+mi}{1000}\PY{l+s+si}{:}\PY{l+s+s1}{.0f}\PY{l+s+si}{\PYZcb{}}\PY{l+s+s1}{ tys.}\PY{l+s+s1}{\PYZsq{}}\PY{p}{)}\PY{p}{)}
\PY{n}{ax}\PY{o}{.}\PY{n}{grid}\PY{p}{(}\PY{k+kc}{True}\PY{p}{,} \PY{n}{axis}\PY{o}{=}\PY{l+s+s1}{\PYZsq{}}\PY{l+s+s1}{y}\PY{l+s+s1}{\PYZsq{}}\PY{p}{,} \PY{n}{linestyle}\PY{o}{=}\PY{l+s+s1}{\PYZsq{}}\PY{l+s+s1}{\PYZhy{}\PYZhy{}}\PY{l+s+s1}{\PYZsq{}}\PY{p}{,} \PY{n}{alpha}\PY{o}{=}\PY{l+m+mf}{0.4}\PY{p}{)}

\PY{c+c1}{\PYZsh{} Etykiety z wartością na końcach słupków}
\PY{k}{for} \PY{n}{p} \PY{o+ow}{in} \PY{n}{ax}\PY{o}{.}\PY{n}{patches}\PY{p}{:}
    \PY{n}{value} \PY{o}{=} \PY{n}{p}\PY{o}{.}\PY{n}{get\PYZus{}height}\PY{p}{(}\PY{p}{)}
    \PY{n}{ax}\PY{o}{.}\PY{n}{annotate}\PY{p}{(}\PY{l+s+sa}{f}\PY{l+s+s1}{\PYZsq{}}\PY{l+s+si}{\PYZob{}}\PY{n}{value}\PY{o}{/}\PY{l+m+mi}{1000}\PY{l+s+si}{:}\PY{l+s+s1}{.1f}\PY{l+s+si}{\PYZcb{}}\PY{l+s+s1}{k}\PY{l+s+s1}{\PYZsq{}}\PY{p}{,}
                \PY{n}{xy}\PY{o}{=}\PY{p}{(}\PY{n}{p}\PY{o}{.}\PY{n}{get\PYZus{}x}\PY{p}{(}\PY{p}{)} \PY{o}{+} \PY{n}{p}\PY{o}{.}\PY{n}{get\PYZus{}width}\PY{p}{(}\PY{p}{)} \PY{o}{/} \PY{l+m+mi}{2}\PY{p}{,} \PY{n}{value}\PY{p}{)}\PY{p}{,}
                \PY{n}{xytext}\PY{o}{=}\PY{p}{(}\PY{l+m+mi}{0}\PY{p}{,} \PY{l+m+mi}{5}\PY{p}{)}\PY{p}{,} \PY{n}{textcoords}\PY{o}{=}\PY{l+s+s1}{\PYZsq{}}\PY{l+s+s1}{offset points}\PY{l+s+s1}{\PYZsq{}}\PY{p}{,}
                \PY{n}{ha}\PY{o}{=}\PY{l+s+s1}{\PYZsq{}}\PY{l+s+s1}{center}\PY{l+s+s1}{\PYZsq{}}\PY{p}{,} \PY{n}{va}\PY{o}{=}\PY{l+s+s1}{\PYZsq{}}\PY{l+s+s1}{bottom}\PY{l+s+s1}{\PYZsq{}}\PY{p}{,} \PY{n}{fontsize}\PY{o}{=}\PY{l+m+mi}{9}\PY{p}{,} \PY{n}{color}\PY{o}{=}\PY{l+s+s1}{\PYZsq{}}\PY{l+s+s1}{black}\PY{l+s+s1}{\PYZsq{}}\PY{p}{)}

\PY{n}{plt}\PY{o}{.}\PY{n}{tight\PYZus{}layout}\PY{p}{(}\PY{p}{)}
\PY{n}{plt}\PY{o}{.}\PY{n}{show}\PY{p}{(}\PY{p}{)}

\PY{c+c1}{\PYZsh{} Wyświetlenie sformatowanej tabeli w notebooku}
\PY{n}{display}\PY{p}{(}\PY{n}{city\PYZus{}stats}\PY{o}{.}\PY{n}{style}\PY{o}{.}\PY{n}{format}\PY{p}{(}\PY{p}{\PYZob{}}\PY{l+s+s1}{\PYZsq{}}\PY{l+s+s1}{mean}\PY{l+s+s1}{\PYZsq{}}\PY{p}{:}\PY{l+s+s1}{\PYZsq{}}\PY{l+s+si}{\PYZob{}:.0f\PYZcb{}}\PY{l+s+s1}{\PYZsq{}}\PY{p}{,} \PY{l+s+s1}{\PYZsq{}}\PY{l+s+s1}{median}\PY{l+s+s1}{\PYZsq{}}\PY{p}{:}\PY{l+s+s1}{\PYZsq{}}\PY{l+s+si}{\PYZob{}:.0f\PYZcb{}}\PY{l+s+s1}{\PYZsq{}}\PY{p}{,}
                                 \PY{l+s+s1}{\PYZsq{}}\PY{l+s+s1}{std}\PY{l+s+s1}{\PYZsq{}}\PY{p}{:}\PY{l+s+s1}{\PYZsq{}}\PY{l+s+si}{\PYZob{}:.0f\PYZcb{}}\PY{l+s+s1}{\PYZsq{}}\PY{p}{,} \PY{l+s+s1}{\PYZsq{}}\PY{l+s+s1}{count}\PY{l+s+s1}{\PYZsq{}}\PY{p}{:}\PY{l+s+s1}{\PYZsq{}}\PY{l+s+si}{\PYZob{}:,.0f\PYZcb{}}\PY{l+s+s1}{\PYZsq{}}\PY{p}{\PYZcb{}}\PY{p}{)}\PY{p}{)}
\end{Verbatim}
\end{tcolorbox}

    \begin{Verbatim}[commandchars=\\\{\}]
C:\textbackslash{}Users\textbackslash{}olale\textbackslash{}AppData\textbackslash{}Local\textbackslash{}Temp\textbackslash{}ipykernel\_12048\textbackslash{}89608967.py:12: FutureWarning:

Passing `palette` without assigning `hue` is deprecated and will be removed in
v0.14.0. Assign the `x` variable to `hue` and set `legend=False` for the same
effect.

  sns.barplot(
    \end{Verbatim}

    \begin{center}
    \adjustimage{max size={0.9\linewidth}{0.9\paperheight}}{Analiza_mieszkan_Final_files/Analiza_mieszkan_Final_28_1.png}
    \end{center}
    { \hspace*{\fill} \\}
    
    
    \begin{Verbatim}[commandchars=\\\{\}]
<pandas.io.formats.style.Styler at 0x1ad18c9d700>
    \end{Verbatim}

    
    \paragraph{Statystyki cen za metr kwadratowy według
miast}\label{statystyki-cen-za-metr-kwadratowy-wedux142ug-miast}

Powyższy wykres prezentuje medianę cen za\,m² w\,15 największych
polskich miastach, opartą na danych z blisko 83 tys. ofert. Najdroższym
rynkiem pozostaje \textbf{Warszawa}, gdzie mediana osiąga \textbf{16,5
tys.\,PLN/m²}, a średnia nawet \textbf{16,9 tys.} Drugi najdroższy rynek
to \textbf{Kraków} (mediana: 14,9\,tys.), a podium zamyka
\textbf{Gdańsk} (13,0\,tys.). Warto zauważyć, że \textbf{różnica między
średnią a medianą} wskazuje na obecność drogich ofert (np. premium lub z
rynku pierwotnego).

Z kolei najtańsze miasta to \textbf{Radom} i \textbf{Częstochowa}, gdzie
mediana oscyluje wokół \textbf{6,3 tys.\,PLN/m²} -- mniej niż
\textbf{40\,\% ceny warszawskiej}.

Wybrane statystyki:

\begin{longtable}[]{@{}
  >{\raggedright\arraybackslash}p{(\linewidth - 8\tabcolsep) * \real{0.1795}}
  >{\raggedright\arraybackslash}p{(\linewidth - 8\tabcolsep) * \real{0.2308}}
  >{\raggedright\arraybackslash}p{(\linewidth - 8\tabcolsep) * \real{0.2308}}
  >{\raggedright\arraybackslash}p{(\linewidth - 8\tabcolsep) * \real{0.1795}}
  >{\raggedright\arraybackslash}p{(\linewidth - 8\tabcolsep) * \real{0.1795}}@{}}
\toprule\noalign{}
\begin{minipage}[b]{\linewidth}\raggedright
Miasto
\end{minipage} & \begin{minipage}[b]{\linewidth}\raggedright
Średnia (PLN/m²)
\end{minipage} & \begin{minipage}[b]{\linewidth}\raggedright
Mediana (PLN/m²)
\end{minipage} & \begin{minipage}[b]{\linewidth}\raggedright
Odch. stand.
\end{minipage} & \begin{minipage}[b]{\linewidth}\raggedright
Liczba ofert
\end{minipage} \\
\midrule\noalign{}
\endhead
\bottomrule\noalign{}
\endlastfoot
Warszawa & 16\,900 & 16\,522 & 3\,877 & 28\,278 \\
Kraków & 15\,338 & 14\,878 & 3\,508 & 12\,542 \\
Gdańsk & 13\,828 & 12\,972 & 3\,608 & 7\,285 \\
Wrocław & 12\,640 & 12\,284 & 2\,581 & 8\,502 \\
Gdynia & 11\,981 & 11\,448 & 3\,110 & 3\,158 \\
Poznań & 10\,749 & 10\,592 & 2\,069 & 3\,246 \\
Rzeszów & 9\,788 & 9\,694 & 1\,971 & 673 \\
Lublin & 9\,349 & 9\,205 & 1\,681 & 2\,037 \\
Białystok & 9\,094 & 8\,925 & 1\,847 & 900 \\
Szczecin & 9\,032 & 8\,828 & 2\,212 & 2\,209 \\
Łódź & 8\,140 & 7\,962 & 1\,600 & 6\,272 \\
Katowice & 8\,392 & 7\,900 & 2\,259 & 2\,127 \\
Bydgoszcz & 7\,719 & 7\,572 & 1\,601 & 3\,231 \\
Radom & 6\,454 & 6\,292 & 1\,061 & 752 \\
Częstochowa & 6\,438 & 6\,285 & 1\,200 & 1\,103 \\
\end{longtable}

\textbf{Odchylenie standardowe} jest szczególnie wysokie w dużych
aglomeracjach (Warszawa, Kraków, Gdańsk), co wskazuje na duży rozrzut
cen.

\textbf{Wnioski:} - Rynek mieszkaniowy w Polsce jest silnie
spolaryzowany -- różnice między najdroższymi a najtańszymi miastami
sięgają ponad \textbf{10 tys. PLN/m²}. - Zróżnicowanie cen sugeruje, że
średnie wartości nie zawsze oddają faktyczny rozkład -- \textbf{mediana
jest bardziej reprezentatywna} w wielu przypadkach. - Najwięcej
analizowanych ofert pochodziło z Warszawy (ponad 28 tys.), co zwiększa
wiarygodność estymacji dla tego miasta. - Warto zauważyć niewilką liczbę
ofert dla Rzeszowa, Białegostoku i Radomia, w związku z czym do wyników
analizy należy podchodzić z lekką rezerwą.

    \begin{tcolorbox}[breakable, size=fbox, boxrule=1pt, pad at break*=1mm,colback=cellbackground, colframe=cellborder]
\prompt{In}{incolor}{43}{\boxspacing}
\begin{Verbatim}[commandchars=\\\{\}]
\PY{c+c1}{\PYZsh{} Analiza 2: Macierz korelacji zmiennych numerycznych}
\PY{c+c1}{\PYZsh{} Macierz korelacji}
\PY{n}{corr\PYZus{}columns} \PY{o}{=} \PY{n}{df}\PY{o}{.}\PY{n}{select\PYZus{}dtypes}\PY{p}{(}\PY{n}{include}\PY{o}{=}\PY{p}{[}\PY{l+s+s1}{\PYZsq{}}\PY{l+s+s1}{number}\PY{l+s+s1}{\PYZsq{}}\PY{p}{]}\PY{p}{)}\PY{o}{.}\PY{n}{columns}
\PY{n}{num\PYZus{}df} \PY{o}{=} \PY{n}{df}\PY{p}{[}\PY{n}{corr\PYZus{}columns}\PY{p}{]}\PY{o}{.}\PY{n}{copy}\PY{p}{(}\PY{p}{)}

\PY{c+c1}{\PYZsh{} Obliczanie macierzy korelacji}
\PY{c+c1}{\PYZsh{} Można użyć różnych metod: \PYZsq{}pearson\PYZsq{}, \PYZsq{}spearman\PYZsq{}, \PYZsq{}kendall}
\PY{n}{corr} \PY{o}{=} \PY{n}{num\PYZus{}df}\PY{o}{.}\PY{n}{corr}\PY{p}{(}\PY{n}{method}\PY{o}{=}\PY{l+s+s1}{\PYZsq{}}\PY{l+s+s1}{pearson}\PY{l+s+s1}{\PYZsq{}}\PY{p}{)}

\PY{c+c1}{\PYZsh{} Tworzenie maski dla górnego trójkąta aby uniknąć duplikacji wartości}
\PY{c+c1}{\PYZsh{} mask = np.triu(np.ones\PYZus{}like(corr, dtype=bool))}

\PY{c+c1}{\PYZsh{} Heatmapa}
\PY{n}{plt}\PY{o}{.}\PY{n}{figure}\PY{p}{(}\PY{n}{figsize}\PY{o}{=}\PY{p}{(}\PY{l+m+mi}{13}\PY{p}{,} \PY{l+m+mi}{10}\PY{p}{)}\PY{p}{)}
\PY{n}{sns}\PY{o}{.}\PY{n}{heatmap}\PY{p}{(}
    \PY{n}{corr}\PY{p}{,}
    \PY{c+c1}{\PYZsh{}mask=mask,}
    \PY{n}{cmap}\PY{o}{=}\PY{l+s+s1}{\PYZsq{}}\PY{l+s+s1}{coolwarm}\PY{l+s+s1}{\PYZsq{}}\PY{p}{,}
    \PY{n}{vmin}\PY{o}{=}\PY{o}{\PYZhy{}}\PY{l+m+mi}{1}\PY{p}{,} \PY{n}{vmax}\PY{o}{=}\PY{l+m+mi}{1}\PY{p}{,}
    \PY{n}{annot}\PY{o}{=}\PY{k+kc}{True}\PY{p}{,} \PY{n}{fmt}\PY{o}{=}\PY{l+s+s2}{\PYZdq{}}\PY{l+s+s2}{.2f}\PY{l+s+s2}{\PYZdq{}}\PY{p}{,}
    \PY{n}{linewidths}\PY{o}{=}\PY{l+m+mf}{0.5}\PY{p}{,}
    \PY{n}{square}\PY{o}{=}\PY{k+kc}{True}\PY{p}{,}
    \PY{n}{cbar\PYZus{}kws}\PY{o}{=}\PY{p}{\PYZob{}}\PY{l+s+s2}{\PYZdq{}}\PY{l+s+s2}{shrink}\PY{l+s+s2}{\PYZdq{}}\PY{p}{:} \PY{l+m+mf}{.8}\PY{p}{\PYZcb{}}
\PY{p}{)}
\PY{n}{plt}\PY{o}{.}\PY{n}{title}\PY{p}{(}\PY{l+s+s1}{\PYZsq{}}\PY{l+s+s1}{Macierz korelacji (Pearson)}\PY{l+s+s1}{\PYZsq{}}\PY{p}{,} \PY{n}{fontsize}\PY{o}{=}\PY{l+m+mi}{16}\PY{p}{,} \PY{n}{pad}\PY{o}{=}\PY{l+m+mi}{12}\PY{p}{)}
\PY{n}{plt}\PY{o}{.}\PY{n}{xticks}\PY{p}{(}\PY{n}{rotation}\PY{o}{=}\PY{l+m+mi}{60}\PY{p}{,} \PY{n}{ha}\PY{o}{=}\PY{l+s+s1}{\PYZsq{}}\PY{l+s+s1}{right}\PY{l+s+s1}{\PYZsq{}}\PY{p}{)}
\PY{n}{plt}\PY{o}{.}\PY{n}{tight\PYZus{}layout}\PY{p}{(}\PY{p}{)}
\PY{n}{plt}\PY{o}{.}\PY{n}{show}\PY{p}{(}\PY{p}{)}
\end{Verbatim}
\end{tcolorbox}

    \begin{center}
    \adjustimage{max size={0.9\linewidth}{0.9\paperheight}}{Analiza_mieszkan_Final_files/Analiza_mieszkan_Final_30_0.png}
    \end{center}
    { \hspace*{\fill} \\}
    
    \paragraph{Macierz korelacji zmiennych
numerycznych}\label{macierz-korelacji-zmiennych-numerycznych}

Najsilniejsze dodatnie zależności:

\begin{itemize}
\tightlist
\item
  \textbf{Metraż (\texttt{squareMeters}) vs.~cena (\texttt{price})}:
  \emph{r}\,=\,0,56\\
  Im większa powierzchnia, tym wyższa cena całkowita --- ten efekt był
  już widoczny na wykresie rozrzutu.
\item
  \textbf{Liczba pokoi (\texttt{rooms}) vs.~metraż}: \emph{r}\,=\,0,8\\
  Więcej pokoi zwykle oznacza większą powierzchnię.
\end{itemize}

Najsilniejsze ujemne zależności:

\begin{itemize}
\tightlist
\item
  \textbf{Rok budowy (\texttt{buildYear}) vs.~odległość od centrum}
  (\texttt{centreDistance}): \emph{r}\,≈\,-0,32\\
  Nowsze inwestycje częściej pojawiają się poza ścisłym centrum.
\item
  \textbf{Odległość od centrum (\texttt{centreDistance}) vs.~liczba POI}
  (\texttt{poiCount}): \emph{r}\,≈\,-0,45
\end{itemize}

    \begin{tcolorbox}[breakable, size=fbox, boxrule=1pt, pad at break*=1mm,colback=cellbackground, colframe=cellborder]
\prompt{In}{incolor}{44}{\boxspacing}
\begin{Verbatim}[commandchars=\\\{\}]
\PY{c+c1}{\PYZsh{} Analiza 2: Wpływ udogodnień na cenę mieszkań}
\PY{n}{amenity\PYZus{}map} \PY{o}{=} \PY{p}{\PYZob{}}
    \PY{l+s+s1}{\PYZsq{}}\PY{l+s+s1}{hasParkingSpace}\PY{l+s+s1}{\PYZsq{}}\PY{p}{:} \PY{l+s+s1}{\PYZsq{}}\PY{l+s+s1}{Miejsce parkingowe}\PY{l+s+s1}{\PYZsq{}}\PY{p}{,}
    \PY{l+s+s1}{\PYZsq{}}\PY{l+s+s1}{hasBalcony}\PY{l+s+s1}{\PYZsq{}}     \PY{p}{:} \PY{l+s+s1}{\PYZsq{}}\PY{l+s+s1}{Balkon}\PY{l+s+s1}{\PYZsq{}}\PY{p}{,}
    \PY{l+s+s1}{\PYZsq{}}\PY{l+s+s1}{hasElevator}\PY{l+s+s1}{\PYZsq{}}    \PY{p}{:} \PY{l+s+s1}{\PYZsq{}}\PY{l+s+s1}{Winda}\PY{l+s+s1}{\PYZsq{}}\PY{p}{,}
    \PY{l+s+s1}{\PYZsq{}}\PY{l+s+s1}{hasSecurity}\PY{l+s+s1}{\PYZsq{}}    \PY{p}{:} \PY{l+s+s1}{\PYZsq{}}\PY{l+s+s1}{Ochrona}\PY{l+s+s1}{\PYZsq{}}\PY{p}{,}
    \PY{l+s+s1}{\PYZsq{}}\PY{l+s+s1}{hasStorageRoom}\PY{l+s+s1}{\PYZsq{}} \PY{p}{:} \PY{l+s+s1}{\PYZsq{}}\PY{l+s+s1}{Komórka lokatorska}\PY{l+s+s1}{\PYZsq{}}
\PY{p}{\PYZcb{}}

\PY{n}{fig}\PY{p}{,} \PY{n}{axes} \PY{o}{=} \PY{n}{plt}\PY{o}{.}\PY{n}{subplots}\PY{p}{(}\PY{l+m+mi}{3}\PY{p}{,} \PY{l+m+mi}{2}\PY{p}{,} \PY{n}{figsize}\PY{o}{=}\PY{p}{(}\PY{l+m+mi}{16}\PY{p}{,} \PY{l+m+mi}{16}\PY{p}{)}\PY{p}{)}
\PY{n}{axes} \PY{o}{=} \PY{n}{axes}\PY{o}{.}\PY{n}{flatten}\PY{p}{(}\PY{p}{)}
\PY{n}{fmt} \PY{o}{=} \PY{n}{ticker}\PY{o}{.}\PY{n}{FuncFormatter}\PY{p}{(}\PY{k}{lambda} \PY{n}{x}\PY{p}{,} \PY{n}{pos}\PY{p}{:} \PY{l+s+sa}{f}\PY{l+s+s1}{\PYZsq{}}\PY{l+s+si}{\PYZob{}}\PY{n}{x}\PY{o}{/}\PY{l+m+mi}{1000}\PY{l+s+si}{:}\PY{l+s+s1}{.0f}\PY{l+s+si}{\PYZcb{}}\PY{l+s+s1}{ tys.}\PY{l+s+s1}{\PYZsq{}}\PY{p}{)}

\PY{c+c1}{\PYZsh{} Wspólny zakres dla osi Y}
\PY{n}{y\PYZus{}min}\PY{p}{,} \PY{n}{y\PYZus{}max} \PY{o}{=} \PY{l+m+mf}{4e5}\PY{p}{,} \PY{l+m+mf}{1.2e6}   \PY{c+c1}{\PYZsh{} 400 k – 1 200 k}

\PY{k}{for} \PY{n}{ax}\PY{p}{,} \PY{p}{(}\PY{n}{col}\PY{p}{,} \PY{n}{title}\PY{p}{)} \PY{o+ow}{in} \PY{n+nb}{zip}\PY{p}{(}\PY{n}{axes}\PY{p}{,} \PY{n}{amenity\PYZus{}map}\PY{o}{.}\PY{n}{items}\PY{p}{(}\PY{p}{)}\PY{p}{)}\PY{p}{:}
    \PY{n}{sns}\PY{o}{.}\PY{n}{boxplot}\PY{p}{(}
        \PY{n}{data}\PY{o}{=}\PY{n}{df}\PY{p}{,}
        \PY{n}{x}\PY{o}{=}\PY{n}{col}\PY{p}{,}
        \PY{n}{y}\PY{o}{=}\PY{l+s+s1}{\PYZsq{}}\PY{l+s+s1}{price}\PY{l+s+s1}{\PYZsq{}}\PY{p}{,}
        \PY{n}{hue}\PY{o}{=}\PY{n}{col}\PY{p}{,}
        \PY{n}{legend}\PY{o}{=}\PY{k+kc}{False}\PY{p}{,}
        \PY{n}{order}\PY{o}{=}\PY{p}{[}\PY{l+s+s1}{\PYZsq{}}\PY{l+s+s1}{no}\PY{l+s+s1}{\PYZsq{}}\PY{p}{,} \PY{l+s+s1}{\PYZsq{}}\PY{l+s+s1}{yes}\PY{l+s+s1}{\PYZsq{}}\PY{p}{]}\PY{p}{,}
        \PY{n}{palette}\PY{o}{=}\PY{l+s+s1}{\PYZsq{}}\PY{l+s+s1}{pastel}\PY{l+s+s1}{\PYZsq{}}\PY{p}{,}
        \PY{n}{ax}\PY{o}{=}\PY{n}{ax}
    \PY{p}{)}
    \PY{n}{ax}\PY{o}{.}\PY{n}{set\PYZus{}title}\PY{p}{(}\PY{l+s+sa}{f}\PY{l+s+s1}{\PYZsq{}}\PY{l+s+s1}{Wpływ }\PY{l+s+si}{\PYZob{}}\PY{n}{title}\PY{o}{.}\PY{n}{lower}\PY{p}{(}\PY{p}{)}\PY{l+s+si}{\PYZcb{}}\PY{l+s+s1}{ na cenę}\PY{l+s+s1}{\PYZsq{}}\PY{p}{)}
    \PY{n}{ax}\PY{o}{.}\PY{n}{set\PYZus{}xlabel}\PY{p}{(}\PY{l+s+sa}{f}\PY{l+s+s1}{\PYZsq{}}\PY{l+s+s1}{Posiada }\PY{l+s+si}{\PYZob{}}\PY{n}{title}\PY{o}{.}\PY{n}{lower}\PY{p}{(}\PY{p}{)}\PY{l+s+si}{\PYZcb{}}\PY{l+s+s1}{\PYZsq{}}\PY{p}{)}
    \PY{n}{ax}\PY{o}{.}\PY{n}{set\PYZus{}ylabel}\PY{p}{(}\PY{l+s+s1}{\PYZsq{}}\PY{l+s+s1}{Cena (PLN)}\PY{l+s+s1}{\PYZsq{}} \PY{k}{if} \PY{n}{col} \PY{o+ow}{in} \PY{p}{[}\PY{l+s+s1}{\PYZsq{}}\PY{l+s+s1}{hasParkingSpace}\PY{l+s+s1}{\PYZsq{}}\PY{p}{,} \PY{l+s+s1}{\PYZsq{}}\PY{l+s+s1}{hasElevator}\PY{l+s+s1}{\PYZsq{}}\PY{p}{,} \PY{l+s+s1}{\PYZsq{}}\PY{l+s+s1}{hasStorage}\PY{l+s+s1}{\PYZsq{}}\PY{p}{]} \PY{k}{else} \PY{l+s+s1}{\PYZsq{}}\PY{l+s+s1}{\PYZsq{}}\PY{p}{)}
    \PY{n}{ax}\PY{o}{.}\PY{n}{set\PYZus{}ylim}\PY{p}{(}\PY{n}{y\PYZus{}min}\PY{p}{,} \PY{n}{y\PYZus{}max}\PY{p}{)}
    \PY{n}{ax}\PY{o}{.}\PY{n}{yaxis}\PY{o}{.}\PY{n}{set\PYZus{}major\PYZus{}formatter}\PY{p}{(}\PY{n}{fmt}\PY{p}{)}
    \PY{n}{ax}\PY{o}{.}\PY{n}{grid}\PY{p}{(}\PY{k+kc}{True}\PY{p}{,} \PY{n}{axis}\PY{o}{=}\PY{l+s+s1}{\PYZsq{}}\PY{l+s+s1}{y}\PY{l+s+s1}{\PYZsq{}}\PY{p}{,} \PY{n}{linestyle}\PY{o}{=}\PY{l+s+s1}{\PYZsq{}}\PY{l+s+s1}{\PYZhy{}\PYZhy{}}\PY{l+s+s1}{\PYZsq{}}\PY{p}{,} \PY{n}{alpha}\PY{o}{=}\PY{l+m+mf}{0.4}\PY{p}{)}

    \PY{c+c1}{\PYZsh{} Mediana jako tekst na wykresie}
    \PY{n}{med} \PY{o}{=} \PY{n}{df}\PY{o}{.}\PY{n}{groupby}\PY{p}{(}\PY{n}{col}\PY{p}{)}\PY{p}{[}\PY{l+s+s1}{\PYZsq{}}\PY{l+s+s1}{price}\PY{l+s+s1}{\PYZsq{}}\PY{p}{]}\PY{o}{.}\PY{n}{median}\PY{p}{(}\PY{p}{)}
    \PY{k}{for} \PY{n}{i}\PY{p}{,} \PY{n}{cat} \PY{o+ow}{in} \PY{n+nb}{enumerate}\PY{p}{(}\PY{p}{[}\PY{l+s+s1}{\PYZsq{}}\PY{l+s+s1}{no}\PY{l+s+s1}{\PYZsq{}}\PY{p}{,} \PY{l+s+s1}{\PYZsq{}}\PY{l+s+s1}{yes}\PY{l+s+s1}{\PYZsq{}}\PY{p}{]}\PY{p}{)}\PY{p}{:}
        \PY{k}{if} \PY{n}{cat} \PY{o+ow}{in} \PY{n}{med}\PY{p}{:}
            \PY{n}{ax}\PY{o}{.}\PY{n}{annotate}\PY{p}{(}\PY{l+s+sa}{f}\PY{l+s+s1}{\PYZsq{}}\PY{l+s+si}{\PYZob{}}\PY{n}{med}\PY{p}{[}\PY{n}{cat}\PY{p}{]}\PY{o}{/}\PY{l+m+mi}{1000}\PY{l+s+si}{:}\PY{l+s+s1}{.0f}\PY{l+s+si}{\PYZcb{}}\PY{l+s+s1}{k}\PY{l+s+s1}{\PYZsq{}}\PY{p}{,}
                        \PY{n}{xy}\PY{o}{=}\PY{p}{(}\PY{n}{i}\PY{p}{,} \PY{n}{med}\PY{p}{[}\PY{n}{cat}\PY{p}{]}\PY{p}{)}\PY{p}{,}
                        \PY{n}{xytext}\PY{o}{=}\PY{p}{(}\PY{l+m+mi}{0}\PY{p}{,} \PY{l+m+mi}{6}\PY{p}{)}\PY{p}{,} \PY{n}{textcoords}\PY{o}{=}\PY{l+s+s1}{\PYZsq{}}\PY{l+s+s1}{offset points}\PY{l+s+s1}{\PYZsq{}}\PY{p}{,}
                        \PY{n}{ha}\PY{o}{=}\PY{l+s+s1}{\PYZsq{}}\PY{l+s+s1}{center}\PY{l+s+s1}{\PYZsq{}}\PY{p}{,} \PY{n}{va}\PY{o}{=}\PY{l+s+s1}{\PYZsq{}}\PY{l+s+s1}{bottom}\PY{l+s+s1}{\PYZsq{}}\PY{p}{,} \PY{n}{fontsize}\PY{o}{=}\PY{l+m+mi}{9}\PY{p}{)}

\PY{c+c1}{\PYZsh{} Ukrycie ostatniego wykresu, jeśli jest nieparzysta liczba udogodnień}
\PY{k}{if} \PY{n+nb}{len}\PY{p}{(}\PY{n}{amenity\PYZus{}map}\PY{p}{)} \PY{o}{\PYZpc{}} \PY{l+m+mi}{2} \PY{o}{!=} \PY{l+m+mi}{0}\PY{p}{:}
    \PY{n}{axes}\PY{p}{[}\PY{o}{\PYZhy{}}\PY{l+m+mi}{1}\PY{p}{]}\PY{o}{.}\PY{n}{axis}\PY{p}{(}\PY{l+s+s1}{\PYZsq{}}\PY{l+s+s1}{off}\PY{l+s+s1}{\PYZsq{}}\PY{p}{)}

\PY{n}{plt}\PY{o}{.}\PY{n}{tight\PYZus{}layout}\PY{p}{(}\PY{p}{)}
\PY{n}{plt}\PY{o}{.}\PY{n}{show}\PY{p}{(}\PY{p}{)}
\end{Verbatim}
\end{tcolorbox}

    \begin{center}
    \adjustimage{max size={0.9\linewidth}{0.9\paperheight}}{Analiza_mieszkan_Final_files/Analiza_mieszkan_Final_32_0.png}
    \end{center}
    { \hspace*{\fill} \\}
    
    \paragraph{Wpływ udogodnień na cenę
mieszkania}\label{wpux142yw-udogodnieux144-na-cenux119-mieszkania}

Powyższe boxploty porównują ceny mieszkań w zależności od posiadania
wybranych udogodnień (miejsca parkingowego, balkonu, windy, komórki
lokatorskiej i ochrony).

\begin{longtable}[]{@{}llll@{}}
\toprule\noalign{}
Udogodnienie & Mediana z udog. & Mediana bez udog. & Różnica \\
\midrule\noalign{}
\endhead
\bottomrule\noalign{}
\endlastfoot
Miejsce parkingowe & \textbf{\textasciitilde749\,tys.\,PLN} &
\textasciitilde650\,tys.\,PLN & +15\,\% \\
Balkon & \textbf{\textasciitilde691\,tys.\,PLN} &
\textasciitilde649\,tys.\,PLN & +6\,\% \\
Winda & \textbf{\textasciitilde720\,tys.\,PLN} &
\textasciitilde622\,tys.\,PLN & +16\,\% \\
Ochrona & \textbf{\textasciitilde800\,tys.\,PLN} &
\textasciitilde659\,tys.\,PLN & +21\,\% \\
Komórka lokatorska & \textbf{\textasciitilde712\,tys.\,PLN} &
\textasciitilde629\,tys.\,PLN & +13\,\% \\
\end{longtable}

Wszystkie analizowane udogodnienia mają pozytywny wpływ na cenę
mieszkań. Największą różnicę cenową obserwujemy dla mieszkań z ochroną,
windą i miejscem parkingowym, co sugeruje, że te cechy są szczególnie
wartościowe dla kupujących. Najmniejszy wpływ na cenę ma balkon.

\textbf{Wniosek:} obecność udogodnień koreluje z wyższą ceną ofertową,
ale skala wpływu różni się w zależności od rodzaju udogodnienia;
czynniki te warto włączyć do modelu predykcyjnego.

    \section{WNIOSKI}\label{wnioski}

    Podsumowując, analiza danych dotyczących cen mieszkań w 15 największych
miastach Polski wykazała silne zróżnicowanie cen -- zarówno ogólnych,
jak i za metr kwadratowy -- ze szczególnie wysokimi wartościami w
Warszawie, Krakowie i Gdańsku. Do najważniejszych czynników wpływających
na cenę należą: powierzchnia mieszkania (najmocniejszy predyktor),
liczba pokoi, obecność udogodnień takich jak winda, miejsce parkingowe
czy ochrona. Analiza korelacji ujawniła, że odległość od centrum ma
mniejszy wpływ niż oczekiwano.

Ograniczeniem badania są dane pochodzące wyłącznie z ogłoszeń (a nie
transakcyjne) i autocenzura cen przez sprzedających. W dalszej pracy
dane te można wykorzystać do budowy predykcyjnego modelu cen mieszkań na
wybranych rynkach lokalnych, uzupełnić o dodatkowe źródła danych lub
przeprowadzić analizy segmentacji rynków.


    % Add a bibliography block to the postdoc
    
    
    
\end{document}
